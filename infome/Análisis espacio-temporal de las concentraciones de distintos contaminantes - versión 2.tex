% Template created in order to write a scientific document from an university or enterprise in spanish.


% We initiate the packages and functions
\documentclass[12pt]{article}
\usepackage{geometry}
\geometry{a4paper,
total={170mm, 257mm},
left=20mm,
top=20mm,
}
\usepackage[utf8]{inputenc}
\usepackage[T1]{fontenc}
\usepackage[spanish]{babel}
\usepackage{mathptmx}
\usepackage{amsmath}
\usepackage{amssymb}
\usepackage{nicefrac}
\usepackage{graphicx}
\usepackage{caption}
\captionsetup[table]{name=Tabla}
\usepackage{float}
\usepackage{subcaption}
\usepackage{hyperref}
\usepackage{xcolor}
\hypersetup{
    colorlinks,
    linkcolor={red!50!black},
    citecolor={blue!50!black},
    urlcolor={blue!80!black}
}
\usepackage{tabularx,booktabs}

\pagenumbering{gobble}


% We begin the document we're writing
\begin{document}

%Title page
\begin{titlepage}

%Corporation's logo
\begin{center}

\begin{minipage}[c]{170mm}
\includegraphics[scale=0.7]{logo_uva}

\hspace{1.2cm}Facultad de ciencias
\end{minipage}

\end{center}
%Title, author and date
\begin{center}

\vspace{4cm}
\Huge \textbf{TRABAJO FIN DE GRADO} \\
\vspace{2cm}
\Large Grado en Física \\
\vspace{2cm}
{\Large \textbf{Análisis espacio-temporal de las concentraciones de distintos contaminantes}} \\
\vspace{4cm}
\normalsize \textbf{Autor: Álvaro Patricio Prieto Pérez} \\
\vspace{1cm}
\textbf{Tutor/es: Isidro Pérez Bartolomé \\ María Ángeles García Pérez}

\end{center}

\end{titlepage}

%Giving thanks
\newpage
\begin{minipage}[]{20cm}
\vspace{10cm}
\end{minipage}
\newpage

\begin{minipage}[]{20cm}
\hspace{8.7cm}
\includegraphics[scale=0.4]{logotipo-3-3}
\vspace{7cm}
\end{minipage}


\begin{flushright}
\normalsize \textit{Introducir agradeciemintos aquí}
\end{flushright}

\newpage

\begin{minipage}[]{20cm}
\vspace{10cm}
\end{minipage}

\newpage

%Table of contents
\parskip = 0.5cm
\newpage
\tableofcontents
\newpage
\pagenumbering{roman}	% We change the page numbering to lowercase roman numbers
\setcounter{page}{1}


%We start the main document from here
\section*{Abstract}
\addcontentsline{toc}{section}{\protect\numberline{}Abstract/Resumen}%

\normalsize Write abstract here.

\vspace{0.5cm}

\section*{Resumen}

\normalsize Introducir resumen aquí.

\newpage

\pagenumbering{arabic}	% We change the page numbering to arabic

\section{Introducción}

\normalsize

\subsection{Antecedentes}

En el Reino Unido, se tienen unos 170 sitios de monitorización horaria de la calidad del aire localizados tanto en zonas rurales y apartadas como en zonas altamente industrializadas y contaminadas. Ahora bien, con estos 170 sitios no es posible dar una estimación certera de la concentración de contaminantes en un sitio cualquiera. Por ello, el modelo de \textit{Air Quality Unified Model} (AQUM) es un modelo de dispersión atmosférica que puede dar una mejor estimación espacial de la concentración de distintos contaminantes, pero que tiene el inconveniente de que los datos obtenidos están sesgados negativamente.

Por este motivo, la \textbf{M}edical \& \textbf{E}nvironmental \textbf{D}ata \textbf{M}ash-up \textbf{I}nfrastructure project (\href{https://www.data-mashup.org.uk/}{\textbf{MEDMI}}) ha estimado las concentraciones de los cuatro contaminantes más perjudiciales para la salud humana, esto es, dióxido de nitrógeno ($NO_{2}$), ozono ($O_{3}$), $PM_{10}$ y $PM_{2,5}$, en el territorio de Inglaterra y Gales para el período del 2007 al 2011 con una resolución espacial de un kilómetro cuadrado. Ello ha sido posible utilizando un modelo espacio-temporal bayesiano para corregir el sesgo que presenta el AQUM. Los datos obtenidos están almacenados en archivos .txt y pueden ser descargados en \href{https://www.data-mashup.org.uk/research-projects/statistical-downscaling-of-gridded-air-quality-data/}{https://www.data-mashup.org.uk/research-projects/statistical-downscaling-of-gridded-air-quality-data/}.

A partir de los datos proporcionados por \textbf{MEDMI}, es posible realizar un estudio de la distribución de la concentración de estos contaminantes en el territorio de Inglaterra y Gales, tanto a escala anual como a escala mensual.

\subsection{Contaminantes}

En esta sección se expondrán las principales características y reacciones que se producen los contaminantes que serán estudiados. Por supuesto, estos contaminantes no son constantes en el tiempo, sino que interactúan entre ellos.

\subsubsection{Óxidos de nitrógeno}

Los óxidos de nitrógeno (denominados $NO_{x}$), esto es, el monóxido de nitrógeno ($NO$) y el dióxido de nitrógeno ($NO_{2}$), juegan un papel importante en la química de la atmósfera. Tienen fuentes naturales, como la intrusión de $NO_{x}$ de la estratosfera, actividad volcánica y bacterial, etc. Ahora bien, la mayor fuente de óxidos de nitrógeno son antropogénicas, principalmente por procesos de combustión, ya sea en fuentes estacionarias (industria, generación de energía, etc.) o móviles (combustión interna de vehículos).

Ahora bien, el dióxido de nitrógeno no es emitido directamente por los procesos de combustión a la atmósfera, pues de los óxidos de nitrógeno emitidos por combustión, el 95\% se corresponde al monóxido de nitrógeno, y el 5\% restante se corresponde al dióxido de nitrógeno. Ahora bien, el monóxido de nitrógeno es capaz de reaccionar con el ozono presente en la troposfera de la siguiente manera:

\begin{equation}
NO + O_{3} \rightarrow NO_{2} + O_{2}
\label{eq:no-to-no2}
\end{equation}

De ese modo, la mayor parte de dióxido de nitrógeno presente en la troposfera no es producto de la emisión directa de procesos de combustión (ya sea industrial, de vehículos, etc.), sino de la reacción con el ozono. Del mismo modo, el dióxido de nitrógeno es capaz de descomponerse con luz solar (con $\lambda \leq 430$ nm, muy próximo al ultravioleta) para volver a formar monóxido de nitrógeno, lo cual permite que pueda formarse ozono, véase:

\begin{equation}
\begin{cases}
NO_{2} + h\upsilon \rightarrow NO + O &\text{($\upsilon \geq 697.67$ THz)}\\
O_{2} + O \rightarrow O_{3} & 
\end{cases} 
\label{eq:no2-to-no}
\end{equation}

De este modo, se establece un equilibrio entre la concentración del $NO$ y la concentración del $NO_{2}$ a través de las reacciones \ref{eq:no-to-no2} y \ref{eq:no2-to-no}. Ahora bien, el dióxido de nitrógeno también puede reaccionar con otros compuestos. Durante el día, el principal sumidero del $NO_{2}$ se produce a través de la reacción con el anión hidróxido\footnote{El anión hidróxido se produce a través de la reacción de un átomo de oxígeno (producido por la descomposición de ozono troposférico debido a rayos UV) con una molécula de agua.}:

\begin{equation}
NO_{2} + OH \rightarrow HNO_{3}
\label{eq:no2-to-hno3}
\end{equation}

Por la noche, en el cual no se producen radicales hidróxidos, el dióxido de nitrógeno es oxidado por el ozono, formando $NO_{3}$. Del mismo modo, este último reacciona con el $NO_{2}$ presente para formar $N_{2}O_{5}$, y este último reacciona con agua para formar ácido nítrico. De ese modo, el tiempo de residencia en la atmósfera del dióxido de nitrógeno es de un día, aproximadamente.

Los impactos de la salud del $NO_{2}$ por sí mismo son difíciles de determinar, pues suele venir acompañado de más contaminantes (la fuentes de $NO_{2}$ son las mismas que gran parte del PM). Ahora bien, se sabe puede afectar en el funcionamiento pulmonar (reduciendo su capacidad), tener impactos en el desarrollo fetal, aumentar la posibilidad de desarrollar cáncer, etc.

\subsubsection{Ozono}

La alta reactividad del ozono cumple un papel importante en la capacidad oxidante de la troposfera, pero la mayor parte del $O_{3}$ presente en la atmósfera se encuentra en la estratosfera: tan sólo un 10\% es ozono troposférico.  Ahora bien, la intrusión de ozono de la estratosfera a la troposfera no es la única fuente de ozono troposférico. Gases como el $NO$ o el $CO$ y compuestos orgánicos (todo ellos emitidos por actividad antropogénica) son capaces de producir ozono por reacciones fotoquímicas.

Por ejemplo, en las reacciones \ref{eq:no-to-no2} y \ref{eq:no2-to-no} se encuentra involucrado el ozono. Dado que el monóxido de nitrógeno y el dióxido de nitrógeno se equilibran rápidamente, no hay producción o creación de ozono en el proceso completo. Ahora bien, dada la alta reactividad de ozono, en atmósferas con una concentración suficiente de $NO$, pueden darse las siguientes reacciones:

\begin{equation}
\begin{cases}
OH + CO + O_{2} \rightarrow HO_{2} + CO_{2} \\
HO_{2} + NO \rightarrow OH + NO_{2} \\
NO_{2} + h\upsilon \rightarrow NO + O \\
O + O_{2} \rightarrow O_{3} \\
\end{cases}
\end{equation}

Podemos ver que en la producción del $NO_{2}$ no ha habido moléculas de ozono involucradas y, del mismo modo, que no ha habido variación en la concentración de $OH$ y del $NO$. Por ello, esta cadena de reacciones genera moléculas de ozono, lo que implica que un aumento de la concentración de $NO$ conlleva un aumento de la concentración de ozono troposférico. Del mismo modo, hay otra serie de reacciones relacionadas con moléculas orgánicas que, en presencia de $NO$, forma $NO_{2}$ y da lugar a producción de ozono por las reacción \ref{eq:no2-to-no}.

Altos niveles de ozono en la troposfera pueden tener impacto en la salud, tanto a corto como a largo plazo. A corto plazo, puede producir alteraciones en el funcionamiento de los aparatos respiratorio y cardiovascular, e incluso elevar la tasa de mortalidad. A largo plazo, puede producir inflamación pulmonar, incentivar el desarrollo de asma y reducir la esperanza de vida. 

\subsubsection*{Partículas \textbf{\texorpdfstring{$PM_{10}$}{PM10}} y \textbf{\texorpdfstring{$PM_{2,5}$}{PM2,5}}}
\addcontentsline{toc}{subsubsection}{\protect\numberline{1.1.3}Partículas $PM_{10}$ y $PM_{2,5}$}%

Las partículas PM (\textit{\textbf{P}articulate \textbf{M}atter}) son una mezcla compleja de diversos componentes, con características físicas y químicas diversas. Debido a su alta heterogeneidad, las diferentes características de PM pueden ser determinantes para distintos efectos de la salud en las personas. Ahora bien, las partículas suelen ser clasificadas por sus propiedades aerodinámicas, pues estas últimas son las que determinan los procesos de disipación y transporte en el aire y, por otro lado, de deposición y limpieza en los conductos respiratorios. Dado que el tamaño de las partículas es crítico en determinar dónde con qué facilidad se depositan en los conductos respiratorios, las partículas PM se clasifican por su diámetro aerodinámico.

El $PM_{10}$ (partículas de menos de 10 $\nicefrac{\mu g}{m^3}$) se forma principalmente de manera mecánica, por la rotura de partículas sólidas, ya sea causado por levantamiento de polvo debido al viento (ya sea en tierra al aire libre, o en procesos de agricultura), por procesos industriales o por procesos biológicos (granos de polen, por ejemplo). Estas partículas son lo suficientemente pequeñas como para penetrar en la región torácica.

El $PM_{2,5}$ (partículas de menos de 2,5 $\nicefrac{\mu g}{m^3}$) es generado principalmente por gases, ya sea por condensación de metales y compuestos orgánicos vaporizados por procesos de combustión, o bien por condensación de gases producto de reacciones químicas en la atmósfera. En particular, la mayor parte del $PM_{2,5}$ procede de sulfatos y sustancias orgánicas. En otras condiciones, los nitratos también pasan a ser los principales precursores, etc. Por ejemplo, a partir del dióxido de nitrógeno, este último puede reaccionar con el agua según la reacción \ref{eq:3}, que forma parte de estas partículas (por ello, una parte del $NO_{2}$ pasa a ozono, y otra se vuelve precursora de $PM_{2,5}$). El $PM_{2,5}$ es lo suficientemente pequeño como para depositarse en los conductos más pequeños del aparato respiratorio y en los alvéolos.

Los efectos del PM en la salud está asociado a un incremento de los síntomas en pacientes con asma, a incrementos en el riesgo de sufrir infartos de miocardio (alteran el aparato circulatorio), inflación en órganos del aparato respiratorio e, incluso, la posibilidad de desarrollar COPD (enfermedad pulmonar obstructiva crónica).

Finalmente, a modo de recuento de los efectos de la salud que tienen los distintos contaminantes, puede decirse que hay evidencias de causar muertes prematuras, ya sea por enfermedades cardiovasculares y enfermedades al aparato respiratorio (incluido cáncer). Del mismo modo, exposiciones a corto y largo plazo pueden reducir la capacidad pulmonar, incrementar las posibilidades de padecer infecciones en el aparato respiratorio y agravar el asma. Por otro lado, las exposiciones de mujeres embarazadas a la contaminación atmosférica puede tener impacto en el desarrollo del feto, así como en el desarrollo de recién nacidos y niños. Además, los impactos de la contaminación atmosférica también dependen de la vulnerabilidad de cada persona: esta última puede aumentar con la edad, con condiciones de deterioro de salud preexistentes, etc.

\subsection{Normativa}

En este apartado se establecerá un resumen de los criterios seguidos por la Organización Mundial de la Salud (\textbf{OMS}) y por la Unión Europea (\textbf{UE}) para la la protección de la salud humana. Tanto el análisis de la distribución de la concentración de contaminantes como clasificación del territorio de Inglaterra y Gales en función de la concentración de contaminantes tendrán en cuenta los criterios de ambos organismos para las medias anuales. Los criterios seguidos por la OMS y por la UE que se tienen en cuenta en este trabajo vienen recogidos en \href{http://www.euro.who.int/en/health-topics/environment-and-health/Housing-and-health/publications/pre-2009/air-quality-guidelines.-global-update-2005.-particulate-matter,-ozone,-nitrogen-dioxide-and-sulfur-dioxide}{Air quality guidelines. Global update 2005. Particulate matter, ozone, nitrogen dioxide and sulfur dioxide} y \href{https://www.boe.es/buscar/doc.php?id=DOUE-L-2008-81053}{Directiva 2008/50/CE del Parlamento Europeo y del Consejo, de 21 de mayo de 2008, relativa a la calidad del aire ambiente y a una atmósfera más limpia en Europa}, respectivamente.

Respecto al $NO_{2}$, tanto la OMS como la UE coinciden en establecer como valor límite la media anual de 40 $\nicefrac{\mu g}{m^{3}}$. Este valor límite está establecido para la protección de la salud pública de los efectos del $NO_{2}$.

Para el $PM_{10}$ y el $PM_{2,5}$, la OMS establece el valor límite de media anual de 20 y 10 $\nicefrac{\mu g}{m^{3}}$, respectivamente; mientras que la UE establece de valor límite la media anual de 40 y 25 $\nicefrac{\mu g}{m^{3}}$, respectivamente. Los valores límites que establece la OMS son los niveles más bajos en los que la mortalidad por cáncer al pulmón y cardiovascular se incrementan por exposición a largo plazo al $PM$.

Finalmente, para el ozono no se establece un valor límite para la media anual, sino un valor límite de la máxima media para medidas octohorarias móviles que no debe superarse por más de un determinado número de días al año. En el caso de la OMS, establece como valor límite la media octohoraria de 100 $\nicefrac{\mu g}{m^{3}}$, mientras que la UE establece como valor límite la media octohoraria de 120 $\nicefrac{\mu g}{m^{3}}$. El valor límite que establece la OMS da una protección adecuada para la salud pública, aunque exposiciones a estas concentraciones pueden aumentar de un 1 al 2 $\%$ la mortalidad diaria.

En la tabla \ref{tab:pollutant_guideline}, se pueden ver los valores límites de la OMS y de la UE para el $NO_{2}$, el $O_{3}$, el $PM_{10}$ y el $PM_{2,5}$.

% Table
\begin{table}[H]
\centering
\begin{tabularx}{\textwidth}{|c| *{2}{>{\centering\arraybackslash}X|}}
\hline
 & \multicolumn{2}{c|}{Valores límite de concentración de contaminantes establecidos por} \\
Contaminante & \multicolumn{2}{c|}{la OMS y la UE} \\ \cline{2-3}
 & Normativa OMS & Normativa UE \\
 \hline
 $NO_{2}$ & 40 $\nicefrac{\mu g}{m^{3}}$ media anual & 40 $\nicefrac{\mu g}{m^{3}}$ media anual \\
 \hline
 $O_{3}$ & 100 $\nicefrac{\mu g}{m^{3}}$ media octohoraria & 120 $\nicefrac{\mu g}{m^{3}}$ media octohoraria \\
 \hline
 $PM_{10}$ & 20 $\nicefrac{\mu g}{m^{3}}$ media anual & 40 $\nicefrac{\mu g}{m^{3}}$ media anual \\
 \hline
 $PM_{2,5}$ & 10 $\nicefrac{\mu g}{m^{3}}$ media anual & 25 $\nicefrac{\mu g}{m^{3}}$ media anual \\
 \hline
\end{tabularx}
\caption{}
\label{tab:pollutant_guideline}
\end{table}

\newpage

\section{Objetivos}

Dado que nosotros partimos de un punto en el que no es necesario realizar las mediciones de la concentración de contaminantes, sino que tomamos los datos proporcionados por MEDMI, en este trabajo nos centraremos en el análisis de estos datos para poder obtener información de la distribución de la concentración de contaminantes en Inglaterra y Gales. Con los datos con la media anual será posible estudiar su distribución año a año, mientras que con los datos con la media diaria podremos tratarlos para hallar la media mensual y, una vez hecho ésto, estudiar la distribución mensual durante un año dado durante el período 2007-2011. Así, podemos establecer como objetivos:

\begin{itemize}
\item Estudiar la distribución espacial de la media mensual de concentración de contaminantes a lo largo de un año, de cara a encontrar patrones en su distribución y explicarlos.

\item Estudiar la distribución espacial de la media anual de concentración de contaminantes de cara a establecer si, en el período 2007-2011, se cumplieron con la normativa de la OMS y de la UE.

\item Estudiar la distribución espacial de la media anual de concentración de contaminantes de cara a encontrar tendencias en su evolución a lo largo del período 2007-2011.
\end{itemize}

\newpage

\section{Datos y metodología}

\subsection{Análisis estadístico}

Antes de utilizar y analizar los datos proporcionados por \textbf{MEDMI}, es necesario hacer un breve repaso de las herramientas estadísticas que pueden utilizarse para analizar distribuciones de datos.

A la hora de observar y estudiar la atmósfera o aspectos de ella, se obtienen numerosos datos de todas las variables que se analizan. Además, se suele estudiar en un período más o menos largo en el tiempo, lo cual hace que haya un conjunto enorme de datos a analizar para poder extraer conclusiones. El análisis estadístico permite tratar, clasificar y entender la información que proporcionan estos datos. Ahora bien, los datos extraídos de la atmósfera no siguen el comportamiento que exigen las asunciones de los métodos estadísticos clásicos (lo que implica que no podamos obtener información fiable aplicando esos métodos), lo cual hace que los métodos que utilicemos sean \textbf{rubustos} y \textbf{resistentes}. Un método \textit{robusto} es aquel que no es sensible a las asunciones acerca de la naturaleza de un conjunto de datos: se desempeña bien en la mayoría de los casos (aunque no de manera óptima); mientras que un método \textit{resistente} es aquel que no se ve influenciado por los extremos de un conjunto de datos.

\subsubsection{Cuantiles}

Supongamos que tenemos un conjunto numérico de datos con $n$ elementos. Un \textbf{cuantil} $q_{p}$ es un número que tiene las mismas unidades que los datos y que los números menores que él son la proporción de los datos que denota el subíndice $p$ (con $0 < p < 1$). Otra forma equivalente de expresar el cuantil $q_{p}$ es mediante el \textbf{percentil}, definido como $p \cdot 100$. Por supuesto, para poder determinar un cuantil o percentil cualquiera, es necesario ordenar de menor a mayor el conjunto de datos  $\left\lbrace x_{1}, x_{2}, x_{3}, x_{4}, ..., x_{n} \right\rbrace$, siendo $ x_{i - 1} \leq x_{i} \leq x_{i + 1}$.

Algunos cuantiles suelen ser bastante utilizados en los análisis de datos, como el cuantil $q_{0,5}$ o mediana (explicado más adelante), o los cuantiles $q_{0,25}$ y $q_{0,75}$ (denominados \textit{cuartiles inferior} y \textit{superior}, respectivamente). Estos tres cuantiles separan el conjunto de datos en cuartos. Del mismo modo, los \textit{terciles inferior} y \textit{superior} ($q_{0,333}$ y $q_{0,667}$, respectivamente) separan al conjunto de datos en tercios. Hay más cuantiles con nombres propios, como los \textit{deciles} ($q_{0,1}$, $q_{0,2}$, ..., $q_{0,9}$), etc.

Con los cuantiles es posible hallar estadísticos sin necesidad de representación gráfica que aportan información acerca de la naturaleza de los datos. A continuación, se expondrán estadísticos que dan información de la \textbf{posición}, \textbf{dispersión}, \textbf{simetría} y \textbf{aplanamiento}.

\subsubsection{Posición}

Los estadísticos que atienden a la posición revelan la tendencia central de los datos estudiados. El estadístico más conocido de esta categoría es la \textbf{media}. Suponiendo que tenemos un conjunto de $n$ datos $\left\lbrace x_{1}, x_{2}, x_{3}, x_{4}, ..., x_{n} \right\rbrace$, la media se expresa como:

\begin{equation}
\hat{x} = \frac{1}{n} \sum^{n}_{i = 1} x_{i}
\label{eq:mean}
\end{equation}

El inconveniente que presenta la media es que no es resistente, pues es muy sensible a los valores extremos en un conjunto de datos. Por ejemplo, los conjuntos $\left\lbrace 10; 12; 9; 7; 8 \right\rbrace$ y $\left\lbrace 2; 1; 3; 2; 38 \right\rbrace$ tienen la misma media ($9,2$). Por ello, se suele utilizar como estadístico robusto y resistente al cuantil $q_{0,5}$, también conocido como \textbf{mediana}. Es el valor central del conjunto de datos: divide el conjunto de datos en dos partes iguales. En caso de que el número de elementos del conjunto de datos sea impar, la mediana es simplemente el elemento en el centro. En caso de que el número de elementos sea par, entonces hay dos valores centrales, por lo que se suele tomar la media de ambos elementos. Así, la mediana puede definirse como:

\begin{equation}
\begin{cases}
q_{0,5} = x_{\nicefrac{(n + 1)}{2}} &\text{si \textit{n} es impar}\\
q_{0,5} = \frac{x_{\frac{n}{2}} + x_{\frac{n}{2} + 1}}{2} &\text{si \textit{n} es par}
\end{cases}
\label{eq:median}
\end{equation}

Si volvemos a los dos conjuntos de datos utilizados de ejemplos anteriormente, vemos que la mediana de cada uno de ellos es distinta: la mediana de $\left\lbrace 10; 12; 9; 7; 8 \right\rbrace$ es $9$, mientras que la mediana de $\left\lbrace 2; 1; 3; 2; 38 \right\rbrace$ es $2$.

\subsubsection{Dispersión}

Los estadísticos que atienden a la dispersión muestran la amplitud de los datos alrededor del centro del conjunto de datos estudiados. El estadístico robusto y resistente más usado y simple es el \textbf{rango intercuatílico} (RIC), que no es más que la diferencia entre el cuartil superior y el cuartil inferior:

\begin{equation}
RIC = q_{0,75} - q_{0,25}
\label{eq:ric}
\end{equation}

Así, podemos ver que el RIC da una idea general de la amplitud en el rango central del conjunto de datos estudiados, y dado que omite los extremos superiores e inferiores, es resistente a datos extremos.

\subsubsection{Simetría}

Los estadísticos que atienden a la simetría revelan la forma en la que los datos están agrupados respecto de su centro. El estadístico robusto y resistente que da cuenta de ésto es el \textbf{índice de Yule-Kendall}, que se define como:

\begin{equation}
\gamma_{YK} = \frac{(q_{0,75} - q_{0,50}) - (q_{0,50} - q_{0,25})}{RIC} = \frac{q_{0,75} - 2 q_{0,50} + q_{0,25}}{RIC}
\label{eq:yule-kendall}
\end{equation}

Así, vemos que el índice de Yule-Kendall expresa la proporción entre, por un lado, la diferencia de la dispersión entre el cuartil superior y la mediana y la mediana y el cuartil inferior y, por otro lado, el RIC. Un conjunto de datos que tenga una simetría perfecta (en el rango central del conjunto de datos), esto es, que $q_{0,75} - q_{0,50} = q_{0,50} - q_{0,25}$, dará $\gamma_{YK} = 0$. Por otro lado, un caso extremo en el que los datos estén agrupados tan a la izquierda que $q_{0,75} = q_{0,50}$, entonces $\gamma_{YK} = -1$. Del mismo modo, en un caso extremo de datos agrupados a la derecha en el que $q_{0,25} = q_{0,50}$, tendremos que $\gamma_{YK} = 1$. Por ello, tenemos que $-1 \leq \gamma_{YK} \leq 1$.

\subsubsection{Aplanamiento}

El aplanamiento da cuenta del peso que tienen las colas superiores e inferiores de un conjunto de datos respecto a su centro. Un estadístico de curtosis que es robusto y resistente es la curtosis robusta, que se define como:

\begin{equation}
K_{r} = \frac{RIC}{2 (q_{0,9} - q_{0,1})}
\label{eq:robust-kurtosis}
\end{equation}

Así, vemos que la curtosis robusta expresa la proporción que existe entre el 50\% de la información en el centro del conjunto de datos (esto es, entre el cuartil inferior y el cuartil superior) y entre la diferencia de los deciles extremos $dz_{0,9}$ y $dz_{0,1}$. Cuanto más cercano a cero sea $K_{r}$, más peso tienen las colas, y viceversa.

Ahora que hemos visto los estadísticos robustos y resistentes que podemos utilizar para el análisis de diversos datos, veamos cómo tratar los datos de los que disponemos. Para cada contaminante que se desea estudiar, se tienen dos tipos de archivos con la estimación de la media de la concentración del contaminante. Uno de ellos atiende a la media anual de la concentración, mientras que el otro tipo de archivo atiende a la media diaria de concentración del contaminante a estudiar. Utilizando ambos tipos de archivos se podrá realizar un análisis exhaustivo de la concentración de contaminantes a lo largo del período 2007-2011.

\subsubsection{Archivos con la media anual de concentración}

Utilizando este tipo de archivos, es posible estudiar la distribución anual de contaminación a lo largo del período del 2007 al 2011 en el que ha sido realizada la estimación. La estructura de datos es la siguiente:

\begin{table}[h]
\centering
\begin{tabularx}{0.4\textwidth}{c *{1}{>{\centering\arraybackslash}X}}
Índice & Concentración \\
1 & $P_{1}$ \\
2 & $P_{2}$ \\
3 & $P_{3}$ \\
$\cdot$ & $\cdot$ \\
$\cdot$ & $\cdot$ \\
$\cdot$ & $\cdot$ \\
n & $P_{n}$ \\
\end{tabularx}
\label{table:ii-1}
\caption{Ejemplo de la estructura de datos de los archivos con la media anual de concentración en un año dado.}
\end{table}

Para cada índice hay asociadas unas coordenadas determinadas (dadas en longitud y latitud) y una concentración $P_{i}$ dadas en $\nicefrac{\mu g}{m^3}$. Con esta estructura de datos, el análisis de distribución de contaminantes alrededor del territorio puede realizarse directamente. Más adelante se dirá la manera de proceder.

\subsubsection{Archivos con la media diaria de contaminación}

Con este tipo de archivos, es posible calcular la media mensual para cada localización y, por ende, poder realizar un estudio de la distribución de la contaminación en Inglaterra y Gales a lo largo de un año dado. La estructura de los datos es la siguiente:

\begin{table}[h]
\centering
\begin{tabularx}{0.5\textwidth}{c *{3}{>{\centering\arraybackslash}X}}
Índice & Mes & Día & Concentración \\
1 & 1 & 1 & $P_{1}$ \\
1 & 1 & 2 & $P_{2}$ \\
$\cdot$ & $\cdot$ & $\cdot$ & $\cdot$ \\
$\cdot$ & $\cdot$ & $\cdot$ & $\cdot$ \\
$\cdot$ & $\cdot$ & $\cdot$ & $\cdot$ \\
1 & 1 & 31 & $P_{31}$ \\
1 & 2 & 1 & $P_{32}$ \\
$\cdot$ & $\cdot$ & $\cdot$ & $\cdot$ \\
$\cdot$ & $\cdot$ & $\cdot$ & $\cdot$ \\
$\cdot$ & $\cdot$ & $\cdot$ & $\cdot$ \\
1 & 12 & 31 & $P_{365}$ \\
2 & 1 & 1 & $P_{366}$ \\
$\cdot$ & $\cdot$ & $\cdot$ & $\cdot$ \\
$\cdot$ & $\cdot$ & $\cdot$ & $\cdot$ \\
$\cdot$ & $\cdot$ & $\cdot$ & $\cdot$ \\
n & 12 & 31 & $P_{365 \cdot n}$ \\
\end{tabularx}
\label{table:ii-2}
\caption{Ejemplo de la estructura de datos de los archivos con la media diaria de concentración en un año dado.}
\end{table}

Como puede verse en la tabla, para cada índice se encuentra, de manera ordenada, la concentración de contaminantes en $\nicefrac{\mu g}{m^3}$ para cada día del año. Sin embargo, para realizar el análisis que hemos descrito, necesitamos una estimación mensual de concentración para cada contaminante. Así, si se implementa un algoritmo que calcula la media mensual de cada localización, se obtiene:

\begin{table}[h]
\centering
\begin{tabularx}{0.5\textwidth}{c *{2}{>{\centering\arraybackslash}X}}
Índice & Mes & Concentración \\
1 & 1 & $\mu_{1}$ \\
1 & 2 & $\mu_{2}$ \\
$\cdot$ & $\cdot$ & $\cdot$\\
$\cdot$ & $\cdot$ & $\cdot$ \\
$\cdot$ & $\cdot$ & $\cdot$ \\
1 & 12 & $\mu_{12}$ \\
2 & 1 & $\mu_{13}$ \\
$\cdot$ & $\cdot$ & $\cdot$ \\
$\cdot$ & $\cdot$ & $\cdot$ \\
$\cdot$ & $\cdot$ & $\cdot$ \\
n & 12 & $\mu_{12 \cdot n}$ \\
\end{tabularx}
\label{table:ii-3}
\caption{Ejemplo de la estructura de datos de los archivos con la media mensual de concentración en un año dado.}
\end{table} 

Ahora, se tiene la media mensual de concentración (representado por ${\mu}_{i}$) en $\nicefrac{\mu g}{m^3}$ para cada índice. Sin embargo, el orden de la tabla no permite ver la distribución a lo largo del territorio para un mes dado. Por ello, hay que aplicar un segundo algoritmo que ordene los datos por el mes, como puede verse en la siguiente tabla:

\begin{table}[h]
\centering
\begin{tabularx}{0.5\textwidth}{c *{2}{>{\centering\arraybackslash}X}}
Índice & Mes & Concentración \\
1 & 1 & $\mu_{1}$ \\
2 & 1 & $\mu_{13}$ \\
$\cdot$ & $\cdot$ & $\cdot$\\
$\cdot$ & $\cdot$ & $\cdot$ \\
$\cdot$ & $\cdot$ & $\cdot$ \\
n & 1 & $\mu_{i}$ \\
1 & 2 & $\mu_{2}$ \\
$\cdot$ & $\cdot$ & $\cdot$ \\
$\cdot$ & $\cdot$ & $\cdot$ \\
$\cdot$ & $\cdot$ & $\cdot$ \\
n & 12 & $\mu_{12 \cdot n}$ \\
\end{tabularx}
\label{table:ii-4}
\caption{Ahora los datos están ordenados mes a mes.}
\end{table}

Con esta nueva estructura en los datos, ya es posible realizar un análisis mes a mes de la distribución de la concentración de contaminantes en el territorio de Inglaterra y Gales en un año dado.

Tanto para los archivos con la media anual de concentración de contaminantes como a los archivos con la media mensual de concentración de contaminantes hemos hallado los estadísticos de posición, dispersión, simetría y aplanamiento que cumplen los requisitos de ser robustos y resistentes, esto es, hemos hallado la mediana, el RIC, el índice de Yule-Kendall y la curtosis robusta (definidas en las ecuaciones \ref{eq:median}, \ref{eq:ric}, \ref{eq:yule-kendall} y \ref{eq:robust-kurtosis}, respectivamente).

\subsection{Análisis gráfico}

Aunque los estadísticos que hemos descrito en los apartados anteriores den información numérica de las características del conjunto de datos que queremos analizar de manera rápida y sencilla de calcular, no dan demasiados detalles ni permiten sacar conclusiones a simple vista, ni es sencillo comparar distintos datos entre sí sólo utilizando estadísticos. Por ello, diversas técnicas de representación de los datos existen para poder, de un vistazo, ver la distribución que tienen los conjuntos de datos y sus características.

\subsubsection{Diagramas de caja y bigotes}

Una de estas herramientas gráficas es el diagrama de caja y bigotes, compuesta por cinco cuantiles que dan información de la distribución del conjunto de datos: el mínimo, el cuartil inferior, la mediana, el cuartil superior y el máximo. En una caja se representan el cuartil inferior, la mediana y el cuartil superior, y lo que se encuentra fuera de ella se representa mediante bigotes. Ahora bien, para tener más detalles en las colas, se suele establecer un límite de representación de los bigotes: todo lo que se halle fuera de ese límite se consideran datos anómalos y se representan individualmente con puntos.

\begin{equation}
\begin{cases}
\text{Límite superior} = q_{0,75} + 1,5 RIC \\
\text{Límite inferior} = q_{0,25} - 1,5 RIC \\
\end{cases}
\label{boxplot_fence}
\end{equation}

Utilizando los datos de media anual de concentración, se han representado diagramas de caja y bigotes con los datos de cada año desde el 2007 hasta el 2011 (de cara a estudiar la evolución de la distribución espacial en el tiempo). Con los datos de media mensual de concentración, se han representado en diagramas de caja y bigotes los datos de cada mes de cada año, de cara a estudiar la evolución de la distribución espacial de la concentración a lo largo de un año). Las figuras \ref{fig:box-ejem-1} y \ref{fig:box-ejem-2} son los diagramas de caja y bigotes de la distribución anual y mensual del $NO_{2}$ a modo de ejemplo, respectivamente.

\begin{figure}[H]
\centering
\begin{subfigure}[H]{0.45\textwidth}
\includegraphics[width=\textwidth]{C:/Users/Faraday/Desktop/practicas_alvaro/images/boxplots/annual/annual_no2_boxplot.png}
\caption{Diagrama de caja y bigotes para cada año del $NO_{2}$}
\label{fig:box-ejem-1}
\end{subfigure}
%
\begin{subfigure}[H]{0.45\textwidth}
\includegraphics[width=\textwidth]{C:/Users/Faraday/Desktop/practicas_alvaro/images/boxplots/monthly/no2_2007_boxplot_definitive.png}
\caption{Diagrama de caja y bigotes de los meses del 2007 para el $NO_{2}$}
\label{fig:box-ejem-2}
\end{subfigure}
\caption{}
\end{figure}

\subsubsection{Histogramas}

Los histogramas representan la cantidad que tiene cierto rango de valores de un conjunto de datos. Para ello, se divide el conjunto de datos en regiones de igual ancho (normalmente) y se cuenta el número de elementos que contiene cada una. Así, el histograma consiste en celdas cuya altura da cuenta del número de elementos que contiene, por lo que aporta información acerca de la distribución de las datos como la posición, dispersión, simetría y aplanamiento de un sólo vistazo. Del mismo modo, es posible ver si la distribución de los datos es unimodal o multimodal rápidamente.

En particular, una cuestión que ayuda mucho de los histogramas es que el área de cada celdilla es proporcional a la probabilidad de obtener un elemento en ese rango de valores (en caso de que el histograma esté normalizado, entonces el área de cada celdilla es la probabilidad de obtener un elemento en ese rango de valores).

Ahora bien, lo complicado a la hora de representar un histograma es escoger la anchura de la celdilla, pues dependiendo de ella la información que se extraiga será útil o no: en caso de ser celdillas demasiado anchas, entonces los detalles importantes de la distribución de los datos no podrás apreciarse; por otro lado, si las celdillas son muy delgadas, entonces el histograma resultante será muy irregular y, por ende, difícil de interpretar. Por ello, una aproximación para establecer el ancho de la celdilla es mediante:

\begin{equation}
anchura = c \frac{RIC}{n^{\nicefrac{1}{3}}}
\label{eq:bin_width}
\end{equation}

Donde \textit{n} es el número total de elementos que tiene nuestro conjunto de datos y c es una constante que puede tomar valor entre 2,0-2,6. Los valores cercanos a 2,6 suelen ser óptimos para distribuciones gaussianas, mientras que los más cercanos a 2,0 son más apropiados para distribuciones asimétricas o multimodales. Hay que tener en cuenta que la anchura que da esta ecuación es una guía, una aproximación al valor óptimo del ancho de la celdilla. El valor óptimo depende de la naturaleza y particularidades de nuestro conjunto de datos.

\subsubsection{Suavizados de histogramas}

Los histogramas pueden dar información muy útil acerca de la distribución de datos que tenemos, aunque tienen algunos inconvenientes. Por ejemplo, dónde colocamos el centro de cada celdilla es un proceso arbitrario (no hay un punto "correcto" donde ponerlos) y, del mismo modo, hay que redondear los valores a los extremos de cada celdilla para ver en qué rango entran. Por ello, una alternativa para el histograma es su suavizado, en el cual no está el inconveniente de tener que redondear los valores de cada elemento y tiene una estructura suave. En los histogramas, cada celda es un "bloque unidad" que se apila junto con otros para poder formar la gráfica. En cambio, en los suavizados se emplean formas características llamadas 'núcleos' que son, en general, más suaves que los rectángulos (en el cuadro \ref{table:nucleos-suavizado} podemos ves los más comunes).

\begin{table}[h]
\centering
\begin{tabularx}{\textwidth}{|c * {2}{>{\centering\arraybackslash}X}}
\hline
\multicolumn{3}{|c|}{Núcleos de suavizado utilizados normalmente} \\
\hline
Nombre & Núcleo $h(t)$ & \multicolumn{1}{c|}{Rango de $t$} \\
\hline
Triangular & $1 - |t|$ & \multicolumn{1}{c|}{$-1 < t < 1$} \\
Rectangular & $\frac{1}{2}$ & \multicolumn{1}{c|}{$-1 < t < 1$} \\
Cuadrático (Epachenikov) & $\frac{3}{4}(1 - t^{2})$ & \multicolumn{1}{c|}{$-1 < t < 1$} \\
Cuártico (Biweight) & $\frac{15}{16}(1 - t^{2})^{2}$ & \multicolumn{1}{c|}{$-1 < t < 1$} \\
Tricúbico & $\frac{70}{81}(1 - |t|^{3})^{3}$ & \multicolumn{1}{c|}{$-1 < t < 1$} \\
Gaussiano & $(2\pi)^{\nicefrac{-1}{2}}exp(\nicefrac{-t^{2}}{2})$ & \multicolumn{1}{c|}{$-\infty < t < \infty$} \\
\hline 
\end{tabularx}
\caption{}
\label{table:nucleos-suavizado}
\end{table}

Así, los análogos a las celdas serían las funciones de suavizado, definidas en la ecuación \ref{eq:funcion-suavizado} como:

\begin{equation}
\hat{f}(x_{0})= \frac{1}{nw}\sum^{n}_{i=1}h\left(\frac{x_{0} - x_{i}}{w}\right)
\label{eq:funcion-suavizado}
\end{equation}

Donde $n$ es el número de elementos de la distribución de datos y $w$ es un parámetro arbitrario.

De este modo, el resultado sería la suma de la altura de todas las funciones de suavizado   (hay tantos funciones de suavizado como elementos tiene la distribución de datos).

A la hora de realizar el suavizado, no es tan importante el núcleo que utilicemos para ello, sino el parámetro $w$, pues en función de este último, el suavizado puede estar 'aplanado' y que no sea posible que apreciemos detalles importantes de la distribución de datos; o al contrario, que tenga muchos picos y detalles y su interpretación resulte difícil. Una buena aproximación inicial para el parámetro $w$ viene definida como:

\begin{equation}
w = \frac{0,9 s}{n^{\nicefrac{1}{5}}}
\label{eq:parametro-w}
\end{equation}

Como ejemplo de histogramas y de sus suavizados, la figura \ref{fig:hist-ejem-1} es el histograma y estimado de densidad de núcleo para la concentración anual de ozono en el año 2007. Claramente, la distribución de datos es multimodal.

\begin{figure}[H]
\centering
\includegraphics[width=0.90\textwidth]{C:/Users/Faraday/Desktop/practicas_alvaro/images/histograms/annual/annual_ozone_2007_histogram.png}
\caption{Histrograma y estimado de densidad de núcleo de la concentración anual de $O_{3}$ en el año 2007}
\label{fig:hist-ejem-1}
\end{figure}

\subsubsection{Mapas}

Mediante interpolación de datos y su suavizado, es posible representar los mapas de concentración de contaminantes, lo cual nos permite visualizar directamente las zonas en las que se da la mayor concentración, así como su distribución espacial a lo largo de Inglaterra y Gales. En las figuras \ref{fig:map-ejem-1} y \ref{fig:map-ejem-2} tenemos dos ejemplos de mapas de concentración del $PM_{10}$, el primero anual y el segundo mensual.

\begin{figure}[H]
\centering
\begin{subfigure}[H]{0.45\textwidth}
\includegraphics[width=\textwidth]{C:/Users/Faraday/Desktop/practicas_alvaro/maps/media/pm10/anual/pm10_2007_anual_mapa.png}
\caption{Mapa de concentración anual del $PM_{10}$ en el año 2007}
\label{fig:map-ejem-1}
\end{subfigure}
%
\begin{subfigure}[H]{0.45\textwidth}
\includegraphics[width=\textwidth]{C:/Users/Faraday/Desktop/practicas_alvaro/maps/media/pm10/2007/pm10_2007_3_mapa.png}
\caption{Mapa de concentración mensual del $PM_{10}$ en marzo del 2007}
\label{fig:map-ejem-2}
\end{subfigure}
\caption{}
\end{figure}

Por otro lado, dado que la resolución de la concentración de cada contaminante es de un kilómetro cuadrado, es posible catalogar el territorio en función de la concentración presente. Ello permite obtener la superficie afectada por distintos intervalos de concentración y puede ayudar a estudiar el número de personas afectadas por ella.

Para representar la concentración de contaminantes en los mapas, se ha utilizado una escala de colores que va del amarillo (valor más bajo) hasta el rojo oscuro (valor más alto). Tanto para los mapas de media de concentración anual como para los mapas de media de concentración mensual, se ha establecido como valor máximo en la escala de colores de cada contaminante el valor límite establecido por la UE para cada uno (salvo para el ozono, cuyo valor límite es la media de concentración de ocho horas).

\subsection{Software complementario}

De cara a realizar este análisis de datos, hay que tener en cuenta que la resolución de concentración de contaminantes es de un kilómetro cuadrado, esto es, cada archivo consta de 151 428 filas. Teniendo en cuenta que hay que analizar cuatro contaminantes en un periodo de cinco años (tanto para la media anual como para la media mensual), ello implica que, en total, hay que clasificar, tratar y analizar más de mil millones de filas. Por supuesto, para implementar los algoritmos descritos anteriormente es necesario recorrer los datos de cara archivo uno a uno; y para el cálculo de estadísticos, representación en histogramas y diagramas de caja y bigotes, etc., es necesario ordenar los datos de cada uno de los archivos de menor a mayor. 

Naturalmente, realizar estas tareas manualmente resultan muy tediosas y complicadas, por no decir imposibles. Utilizando programas como Excel y otros, es posible realizar todas estas operaciones, pero aún así resulta muy complicado y lento aplicar cada cuestión que hemos mencionado a cada archivo (que además, debido al gran tamaño que tienen, estos programas no son capaces de abrir). Es por ello que se ha escrito un código en \textit{Python 3.7} que se encarga de abrir, tratar, calcular los estadísticos y realizar todas las gráficas comentadas anteriormente para cada uno de los archivos de los contaminantes que se desean estudiar. Ello permite que todas estas cuestiones se realicen de manera mucho más rápida que si se hiciera 'manualmente' utilizando un programa externo.

Por otro lado, se ha utilizado el programa \textit{Surfer} para poder representar los mapas de concentración de contaminantes de Inglaterra y Gales. Este programa se encargaba de la interpolación y el suavizado de los datos de los que disponíamos.

\newpage

\section{Resultados}

En este apartado serán presentados las gráficas, estadísticos e información obtenidas mediante el código escrito en Python 3.7. Del mismo modo, se intentará explicar la distribución espacial de la concentración de los distintos contaminantes y, de ser posible, ver cuál es la tendencia para cada uno de ellos.

\subsection{Dióxido de nitrógeno}

Como vimos en apartados anteriores, la mayor parte del $NO_{2}$ no es emitido directamente a la atmósfera, sino que es producido en su mayor parte mediante la reacción del $NO$ con el $O_{3}$, como puede verse en la reacción \ref{eq:no-to-no2}. Ahora bien, el ozono puede disociarse descomponerse con luz solar y reaccionar con el $NO_{2}$ para volver a formar $NO$, como puede verse en la reacción \ref{eq:no2-to-no}.

Por ello, la explicación de lo que ocurre será propuesta al final del apartado del ozono, cuando se haya analizado la información de ambos contaminantes. Dado que Inglaterra y Gales son territorios muy húmedos, una parte del $NO_{2}$ puede reaccionar con los aniones hidróxido según la reacción \ref{eq:no2-to-hno3}, por lo que habrá que tenerlo en cuenta a la hora de analizar la interrelación entre el ozono y el dióxido de nitrógeno.

Empezando con los diagramas de caja y bigotes con la media mensual de concentración del dióxido de nitrógeno (figura \ref{fig:box-no2-monthly}), vemos en todos los años que la máxima concentración se da en meses fríos de otoño e invierno (sobre todo febrero y noviembre, aunque hay años en los que también hay mayor concentración en enero, diciembre e incluso en octubre), mientras que la concentración mínima se da en meses cálidos de verano (sobre todo en junio, julio y agosto). Dependiendo del año, estas diferencias son más o menos pronunciadas, pero en todos los años del período analizado, se cumple esta regla

% Monthly boxplots
\begin{figure}[H]
\centering
\begin{subfigure}[H]{0.30\textwidth}
\includegraphics[width=\textwidth]{C:/Users/Faraday/Desktop/practicas_alvaro/images/boxplots/monthly/no2_2007_boxplot_definitive.png}
\captionsetup{labelformat=empty}
\caption{2007}
\label{fig:box-no2-2007}
\end{subfigure}
%
\begin{subfigure}[H]{0.30\textwidth}
\includegraphics[width=\textwidth]{C:/Users/Faraday/Desktop/practicas_alvaro/images/boxplots/monthly/no2_2008_boxplot_definitive.png}
\captionsetup{labelformat=empty}
\caption{2008}
\label{fig:box-no2-2008}
\end{subfigure}
%
\begin{subfigure}[H]{0.30\textwidth}
\includegraphics[width=\textwidth]{C:/Users/Faraday/Desktop/practicas_alvaro/images/boxplots/monthly/no2_2009_boxplot_definitive.png}
\captionsetup{labelformat=empty}
\caption{2009}
\label{fig:box-no2-2009}
\end{subfigure}

\begin{subfigure}[H]{0.30\textwidth}
\includegraphics[width=\textwidth]{C:/Users/Faraday/Desktop/practicas_alvaro/images/boxplots/monthly/no2_2010_boxplot_definitive.png}
\captionsetup{labelformat=empty}
\caption{2010}
\label{fig:box-no2-2010}
\end{subfigure}
%
\begin{subfigure}[H]{0.30\textwidth}
\includegraphics[width=\textwidth]{C:/Users/Faraday/Desktop/practicas_alvaro/images/boxplots/monthly/no2_2011_boxplot_definitive.png}
\captionsetup{labelformat=empty}
\caption{2011}
\label{fig:box-no2-2011}
\end{subfigure}
\caption{Diagramas de caja y bigotes de la media mensual concentración de $NO_{2}$ en el período 2007-2011}
\label{fig:box-no2-monthly}
\end{figure}

Ahora, si se analizan los mapas de concentración de dióxido de nitrógeno (figuras \ref{fig:map-no2-2007}, \ref{fig:map-no2-2008}, \ref{fig:map-no2-2009}, \ref{fig:map-no2-2010} y \ref{fig:map-no2-2011}) es posible determinar la distribución de la concentración de $NO_{2}$ en Inglaterra y Gales.

Las zonas más contaminadas a lo largo del año son las ciudades de Londres, Oxford, Brístol y Cardiff (localizadas al sur de Inglaterra y Gales), la ciudad de Birmingham (en el centro de Inglaterra), la ciudad de Mánchester (localizada en la costa oeste de Inglaterra) y las ciudades de Kingston Upon Hill y Sunderland (al noreste de Inglaterra). Alrededor de estas ciudades, la concentración de $NO_{2}$ va disminuyendo progresivamente. Sin embargo, en los meses de mayor concentración (febrero, noviembre y diciembre) puede que se superen los 40 $\nicefrac{\mu g}{m^3}$ (el valor límite de media anual establecida por la OMS y la UE, tabla \ref{tab:pollutant_guideline}) en zonas muy extensas del territorio estudiado, ya sea el sur de Inglaterra o, directamente, la mayor parte de este mismo país.

Por otro lado, las zonas menos contaminadas con diferencia son el oeste de Gales y el norte de Inglaterra (con excepción de las ciudades de Sunderland y Kingston Upon Hill). En los meses con mayor concentración de $NO_{2}$ también aumentó aquí la concentración, pero nunca de la manera en la que se concentraba en otras zonas del territorio.

% We include the graphics. Year 2007
\begin{figure}[H]
\centering
\begin{subfigure}[H]{0.15\textwidth}
\includegraphics[width=\textwidth]{C:/Users/Faraday/Desktop/practicas_alvaro/maps/media/no2/2007/no2_2007_1_mapa.png}
\captionsetup{labelformat=empty}
\caption{Enero}
\label{fig:map-no2-2007-1}
\end{subfigure}
%
\begin{subfigure}[H]{0.15\textwidth}
\includegraphics[width=\textwidth]{C:/Users/Faraday/Desktop/practicas_alvaro/maps/media/no2/2007/no2_2007_2_mapa.png}
\captionsetup{labelformat=empty}
\caption{Febrero}
\label{fig:map-no2-2007-2}
\end{subfigure}
%
\begin{subfigure}[H]{0.15\textwidth}
\includegraphics[width=\textwidth]{C:/Users/Faraday/Desktop/practicas_alvaro/maps/media/no2/2007/no2_2007_3_mapa.png}
\captionsetup{labelformat=empty}
\caption{Marzo}
\label{fig:map-no2-2007-3}
\end{subfigure}
%
\begin{subfigure}[H]{0.15\textwidth}
\includegraphics[width=\textwidth]{C:/Users/Faraday/Desktop/practicas_alvaro/maps/media/no2/2007/no2_2007_4_mapa.png}
\captionsetup{labelformat=empty}
\caption{Abril}
\label{fig:map-no2-2007-4}
\end{subfigure}
%
\begin{subfigure}[H]{0.15\textwidth}
\includegraphics[width=\textwidth]{C:/Users/Faraday/Desktop/practicas_alvaro/maps/media/no2/2007/no2_2007_5_mapa.png}
\captionsetup{labelformat=empty}
\caption{Mayo}
\label{fig:map-no2-2007-5}
\end{subfigure}
%
\begin{subfigure}[H]{0.15\textwidth}
\includegraphics[width=\textwidth]{C:/Users/Faraday/Desktop/practicas_alvaro/maps/media/no2/2007/no2_2007_6_mapa.png}
\captionsetup{labelformat=empty}
\caption{Junio}
\label{fig:map-no2-2007-6}
\end{subfigure}

\begin{subfigure}[H]{0.15\textwidth}
\includegraphics[width=\textwidth]{C:/Users/Faraday/Desktop/practicas_alvaro/maps/media/no2/2007/no2_2007_7_mapa.png}
\captionsetup{labelformat=empty}
\caption{Julio}
\label{fig:map-no2-2007-7}
\end{subfigure}
%
\begin{subfigure}[H]{0.15\textwidth}
\includegraphics[width=\textwidth]{C:/Users/Faraday/Desktop/practicas_alvaro/maps/media/no2/2007/no2_2007_8_mapa.png}
\captionsetup{labelformat=empty}
\caption{Agosto}
\label{fig:map-no2-2007-8}
\end{subfigure}
%
\begin{subfigure}[H]{0.15\textwidth}
\includegraphics[width=\textwidth]{C:/Users/Faraday/Desktop/practicas_alvaro/maps/media/no2/2007/no2_2007_9_mapa.png}
\captionsetup{labelformat=empty}
\caption{Septiembre}
\label{fig:map-no2-2007-9}
\end{subfigure}
%
\begin{subfigure}[H]{0.15\textwidth}
\includegraphics[width=\textwidth]{C:/Users/Faraday/Desktop/practicas_alvaro/maps/media/no2/2007/no2_2007_10_mapa.png}
\captionsetup{labelformat=empty}
\caption{Octubre}
\label{fig:map-no2-2007-10}
\end{subfigure}
%
\begin{subfigure}[H]{0.15\textwidth}
\includegraphics[width=\textwidth]{C:/Users/Faraday/Desktop/practicas_alvaro/maps/media/no2/2007/no2_2007_11_mapa.png}
\captionsetup{labelformat=empty}
\caption{Noviembre}
\label{fig:map-no2-2007-11}
\end{subfigure}
%
\begin{subfigure}[H]{0.15\textwidth}
\includegraphics[width=\textwidth]{C:/Users/Faraday/Desktop/practicas_alvaro/maps/media/no2/2007/no2_2007_12_mapa.png}
\captionsetup{labelformat=empty}
\caption{Diciembre}
\label{fig:map-no2-2007-12}
\end{subfigure}

\begin{subfigure}[H]{0.45\textwidth}
\includegraphics[width=\textwidth]{C:/Users/Faraday/Desktop/practicas_alvaro/maps/media/no2/color_scale.png}
\captionsetup{labelformat=empty}
\caption{}
\end{subfigure}

\vspace*{-7mm}
\caption{Media mensual de concentración del $NO_{2}$ en el año 2007}
\label{fig:map-no2-2007}
\end{figure}

% We include the graphics. Year 2008
\begin{figure}[H]
\centering
\begin{subfigure}[H]{0.15\textwidth}
\includegraphics[width=\textwidth]{C:/Users/Faraday/Desktop/practicas_alvaro/maps/media/no2/2008/no2_2008_1_mapa.png}
\captionsetup{labelformat=empty}
\caption{Enero}
\label{fig:map-no2-2008-1}
\end{subfigure}
%
\begin{subfigure}[H]{0.15\textwidth}
\includegraphics[width=\textwidth]{C:/Users/Faraday/Desktop/practicas_alvaro/maps/media/no2/2008/no2_2008_2_mapa.png}
\captionsetup{labelformat=empty}
\caption{Febrero}
\label{fig:map-no2-2008-2}
\end{subfigure}
%
\begin{subfigure}[H]{0.15\textwidth}
\includegraphics[width=\textwidth]{C:/Users/Faraday/Desktop/practicas_alvaro/maps/media/no2/2008/no2_2008_3_mapa.png}
\captionsetup{labelformat=empty}
\caption{Marzo}
\label{fig:map-no2-2008-3}
\end{subfigure}
%
\begin{subfigure}[H]{0.15\textwidth}
\includegraphics[width=\textwidth]{C:/Users/Faraday/Desktop/practicas_alvaro/maps/media/no2/2008/no2_2008_4_mapa.png}
\captionsetup{labelformat=empty}
\caption{Abril}
\label{fig:map-no2-2008-4}
\end{subfigure}
%
\begin{subfigure}[H]{0.15\textwidth}
\includegraphics[width=\textwidth]{C:/Users/Faraday/Desktop/practicas_alvaro/maps/media/no2/2008/no2_2008_5_mapa.png}
\captionsetup{labelformat=empty}
\caption{Mayo}
\label{fig:map-no2-2008-5}
\end{subfigure}
%
\begin{subfigure}[H]{0.15\textwidth}
\includegraphics[width=\textwidth]{C:/Users/Faraday/Desktop/practicas_alvaro/maps/media/no2/2008/no2_2008_6_mapa.png}
\captionsetup{labelformat=empty}
\caption{Junio}
\label{fig:map-no2-2008-6}
\end{subfigure}

\begin{subfigure}[H]{0.15\textwidth}
\includegraphics[width=\textwidth]{C:/Users/Faraday/Desktop/practicas_alvaro/maps/media/no2/2008/no2_2008_7_mapa.png}
\captionsetup{labelformat=empty}
\caption{Julio}
\label{fig:map-no2-2008-7}
\end{subfigure}
%
\begin{subfigure}[H]{0.15\textwidth}
\includegraphics[width=\textwidth]{C:/Users/Faraday/Desktop/practicas_alvaro/maps/media/no2/2008/no2_2008_8_mapa.png}
\captionsetup{labelformat=empty}
\caption{Agosto}
\label{fig:map-no2-2008-8}
\end{subfigure}
%
\begin{subfigure}[H]{0.15\textwidth}
\includegraphics[width=\textwidth]{C:/Users/Faraday/Desktop/practicas_alvaro/maps/media/no2/2008/no2_2008_9_mapa.png}
\captionsetup{labelformat=empty}
\caption{Septiembre}
\label{fig:map-no2-2008-9}
\end{subfigure}
%
\begin{subfigure}[H]{0.15\textwidth}
\includegraphics[width=\textwidth]{C:/Users/Faraday/Desktop/practicas_alvaro/maps/media/no2/2008/no2_2008_10_mapa.png}
\captionsetup{labelformat=empty}
\caption{Octubre}
\label{fig:map-no2-2008-10}
\end{subfigure}
%
\begin{subfigure}[H]{0.15\textwidth}
\includegraphics[width=\textwidth]{C:/Users/Faraday/Desktop/practicas_alvaro/maps/media/no2/2008/no2_2008_11_mapa.png}
\captionsetup{labelformat=empty}
\caption{Noviembre}
\label{fig:map-no2-2008-11}
\end{subfigure}
%
\begin{subfigure}[H]{0.15\textwidth}
\includegraphics[width=\textwidth]{C:/Users/Faraday/Desktop/practicas_alvaro/maps/media/no2/2008/no2_2008_12_mapa.png}
\captionsetup{labelformat=empty}
\caption{Diciembre}
\label{fig:map-no2-2008-12}
\end{subfigure}

\begin{subfigure}[H]{0.45\textwidth}
\includegraphics[width=\textwidth]{C:/Users/Faraday/Desktop/practicas_alvaro/maps/media/no2/color_scale.png}
\captionsetup{labelformat=empty}
\caption{}
\end{subfigure}

\vspace*{-7mm}
\caption{Media mensual de concentración del $NO_{2}$ en el año 2008}
\label{fig:map-no2-2008}
\end{figure}

% We include the graphics. Year 2009
\begin{figure}[H]
\centering
\begin{subfigure}[H]{0.15\textwidth}
\includegraphics[width=\textwidth]{C:/Users/Faraday/Desktop/practicas_alvaro/maps/media/no2/2009/no2_2009_1_mapa.png}
\captionsetup{labelformat=empty}
\caption{Enero}
\label{fig:map-no2-2009-1}
\end{subfigure}
%
\begin{subfigure}[H]{0.15\textwidth}
\includegraphics[width=\textwidth]{C:/Users/Faraday/Desktop/practicas_alvaro/maps/media/no2/2009/no2_2009_2_mapa.png}
\captionsetup{labelformat=empty}
\caption{Febrero}
\label{fig:map-no2-2009-2}
\end{subfigure}
%
\begin{subfigure}[H]{0.15\textwidth}
\includegraphics[width=\textwidth]{C:/Users/Faraday/Desktop/practicas_alvaro/maps/media/no2/2009/no2_2009_3_mapa.png}
\captionsetup{labelformat=empty}
\caption{Marzo}
\label{fig:map-no2-2009-3}
\end{subfigure}
%
\begin{subfigure}[H]{0.15\textwidth}
\includegraphics[width=\textwidth]{C:/Users/Faraday/Desktop/practicas_alvaro/maps/media/no2/2009/no2_2009_4_mapa.png}
\captionsetup{labelformat=empty}
\caption{Abril}
\label{fig:map-no2-2009-4}
\end{subfigure}
%
\begin{subfigure}[H]{0.15\textwidth}
\includegraphics[width=\textwidth]{C:/Users/Faraday/Desktop/practicas_alvaro/maps/media/no2/2009/no2_2009_5_mapa.png}
\captionsetup{labelformat=empty}
\caption{Mayo}
\label{fig:map-no2-2009-5}
\end{subfigure}
%
\begin{subfigure}[H]{0.15\textwidth}
\includegraphics[width=\textwidth]{C:/Users/Faraday/Desktop/practicas_alvaro/maps/media/no2/2009/no2_2009_6_mapa.png}
\captionsetup{labelformat=empty}
\caption{Junio}
\label{fig:map-no2-2009-6}
\end{subfigure}

\begin{subfigure}[H]{0.15\textwidth}
\includegraphics[width=\textwidth]{C:/Users/Faraday/Desktop/practicas_alvaro/maps/media/no2/2009/no2_2009_7_mapa.png}
\captionsetup{labelformat=empty}
\caption{Julio}
\label{fig:map-no2-2009-7}
\end{subfigure}
%
\begin{subfigure}[H]{0.15\textwidth}
\includegraphics[width=\textwidth]{C:/Users/Faraday/Desktop/practicas_alvaro/maps/media/no2/2009/no2_2009_8_mapa.png}
\captionsetup{labelformat=empty}
\caption{Agosto}
\label{fig:map-no2-2009-8}
\end{subfigure}
%
\begin{subfigure}[H]{0.15\textwidth}
\includegraphics[width=\textwidth]{C:/Users/Faraday/Desktop/practicas_alvaro/maps/media/no2/2009/no2_2009_9_mapa.png}
\captionsetup{labelformat=empty}
\caption{Septiembre}
\label{fig:map-no2-2009-9}
\end{subfigure}
%
\begin{subfigure}[H]{0.15\textwidth}
\includegraphics[width=\textwidth]{C:/Users/Faraday/Desktop/practicas_alvaro/maps/media/no2/2009/no2_2009_10_mapa.png}
\captionsetup{labelformat=empty}
\caption{Octubre}
\label{fig:map-no2-2009-10}
\end{subfigure}
%
\begin{subfigure}[H]{0.15\textwidth}
\includegraphics[width=\textwidth]{C:/Users/Faraday/Desktop/practicas_alvaro/maps/media/no2/2009/no2_2009_11_mapa.png}
\captionsetup{labelformat=empty}
\caption{Noviembre}
\label{fig:map-no2-2009-11}
\end{subfigure}
%
\begin{subfigure}[H]{0.15\textwidth}
\includegraphics[width=\textwidth]{C:/Users/Faraday/Desktop/practicas_alvaro/maps/media/no2/2009/no2_2009_12_mapa.png}
\captionsetup{labelformat=empty}
\caption{Diciembre}
\label{fig:map-no2-2009-12}
\end{subfigure}

\begin{subfigure}[H]{0.45\textwidth}
\includegraphics[width=\textwidth]{C:/Users/Faraday/Desktop/practicas_alvaro/maps/media/no2/color_scale.png}
\captionsetup{labelformat=empty}
\caption{}
\end{subfigure}

\vspace*{-7mm}
\caption{Media mensual de concentración del $NO_{2}$ en el año 2009}
\label{fig:map-no2-2009}
\end{figure}

% We include the graphics. Year 2010
\begin{figure}[H]
\centering
\begin{subfigure}[H]{0.15\textwidth}
\includegraphics[width=\textwidth]{C:/Users/Faraday/Desktop/practicas_alvaro/maps/media/no2/2010/no2_2010_1_mapa.png}
\captionsetup{labelformat=empty}
\caption{Enero}
\label{fig:map-no2-2010-1}
\end{subfigure}
%
\begin{subfigure}[H]{0.15\textwidth}
\includegraphics[width=\textwidth]{C:/Users/Faraday/Desktop/practicas_alvaro/maps/media/no2/2010/no2_2010_2_mapa.png}
\captionsetup{labelformat=empty}
\caption{Febrero}
\label{fig:map-no2-2010-2}
\end{subfigure}
%
\begin{subfigure}[H]{0.15\textwidth}
\includegraphics[width=\textwidth]{C:/Users/Faraday/Desktop/practicas_alvaro/maps/media/no2/2010/no2_2010_3_mapa.png}
\captionsetup{labelformat=empty}
\caption{Marzo}
\label{fig:map-no2-2010-3}
\end{subfigure}
%
\begin{subfigure}[H]{0.15\textwidth}
\includegraphics[width=\textwidth]{C:/Users/Faraday/Desktop/practicas_alvaro/maps/media/no2/2010/no2_2010_4_mapa.png}
\captionsetup{labelformat=empty}
\caption{Abril}
\label{fig:map-no2-2010-4}
\end{subfigure}
%
\begin{subfigure}[H]{0.15\textwidth}
\includegraphics[width=\textwidth]{C:/Users/Faraday/Desktop/practicas_alvaro/maps/media/no2/2010/no2_2010_5_mapa.png}
\captionsetup{labelformat=empty}
\caption{Mayo}
\label{fig:map-no2-2010-5}
\end{subfigure}
%
\begin{subfigure}[H]{0.15\textwidth}
\includegraphics[width=\textwidth]{C:/Users/Faraday/Desktop/practicas_alvaro/maps/media/no2/2010/no2_2010_6_mapa.png}
\captionsetup{labelformat=empty}
\caption{Junio}
\label{fig:map-no2-2010-6}
\end{subfigure}

\begin{subfigure}[H]{0.15\textwidth}
\includegraphics[width=\textwidth]{C:/Users/Faraday/Desktop/practicas_alvaro/maps/media/no2/2010/no2_2010_7_mapa.png}
\captionsetup{labelformat=empty}
\caption{Julio}
\label{fig:map-no2-2010-7}
\end{subfigure}
%
\begin{subfigure}[H]{0.15\textwidth}
\includegraphics[width=\textwidth]{C:/Users/Faraday/Desktop/practicas_alvaro/maps/media/no2/2010/no2_2010_8_mapa.png}
\captionsetup{labelformat=empty}
\caption{Agosto}
\label{fig:map-no2-2010-8}
\end{subfigure}
%
\begin{subfigure}[H]{0.15\textwidth}
\includegraphics[width=\textwidth]{C:/Users/Faraday/Desktop/practicas_alvaro/maps/media/no2/2010/no2_2010_9_mapa.png}
\captionsetup{labelformat=empty}
\caption{Septiembre}
\label{fig:map-no2-2010-9}
\end{subfigure}
%
\begin{subfigure}[H]{0.15\textwidth}
\includegraphics[width=\textwidth]{C:/Users/Faraday/Desktop/practicas_alvaro/maps/media/no2/2010/no2_2010_10_mapa.png}
\captionsetup{labelformat=empty}
\caption{Octubre}
\label{fig:map-no2-2010-10}
\end{subfigure}
%
\begin{subfigure}[H]{0.15\textwidth}
\includegraphics[width=\textwidth]{C:/Users/Faraday/Desktop/practicas_alvaro/maps/media/no2/2010/no2_2010_11_mapa.png}
\captionsetup{labelformat=empty}
\caption{Noviembre}
\label{fig:map-no2-2010-11}
\end{subfigure}
%
\begin{subfigure}[H]{0.15\textwidth}
\includegraphics[width=\textwidth]{C:/Users/Faraday/Desktop/practicas_alvaro/maps/media/no2/2010/no2_2010_12_mapa.png}
\captionsetup{labelformat=empty}
\caption{Diciembre}
\label{fig:map-no2-2010-12}
\end{subfigure}

\begin{subfigure}[H]{0.45\textwidth}
\includegraphics[width=\textwidth]{C:/Users/Faraday/Desktop/practicas_alvaro/maps/media/no2/color_scale.png}
\captionsetup{labelformat=empty}
\caption{}
\end{subfigure}

\vspace*{-7mm}
\caption{Media mensual de concentración del $NO_{2}$ en el año 2010}
\label{fig:map-no2-2010}
\end{figure}

% We include the graphics. Year 2011
\begin{figure}[H]
\centering
\begin{subfigure}[H]{0.15\textwidth}
\includegraphics[width=\textwidth]{C:/Users/Faraday/Desktop/practicas_alvaro/maps/media/no2/2011/no2_2011_1_mapa.png}
\captionsetup{labelformat=empty}
\caption{Enero}
\label{fig:map-no2-2011-1}
\end{subfigure}
%
\begin{subfigure}[H]{0.15\textwidth}
\includegraphics[width=\textwidth]{C:/Users/Faraday/Desktop/practicas_alvaro/maps/media/no2/2011/no2_2011_2_mapa.png}
\captionsetup{labelformat=empty}
\caption{Febrero}
\label{fig:map-no2-2011-2}
\end{subfigure}
%
\begin{subfigure}[H]{0.15\textwidth}
\includegraphics[width=\textwidth]{C:/Users/Faraday/Desktop/practicas_alvaro/maps/media/no2/2011/no2_2011_3_mapa.png}
\captionsetup{labelformat=empty}
\caption{Marzo}
\label{fig:map-no2-2011-3}
\end{subfigure}
%
\begin{subfigure}[H]{0.15\textwidth}
\includegraphics[width=\textwidth]{C:/Users/Faraday/Desktop/practicas_alvaro/maps/media/no2/2011/no2_2011_4_mapa.png}
\captionsetup{labelformat=empty}
\caption{Abril}
\label{fig:map-no2-2011-4}
\end{subfigure}
%
\begin{subfigure}[H]{0.15\textwidth}
\includegraphics[width=\textwidth]{C:/Users/Faraday/Desktop/practicas_alvaro/maps/media/no2/2011/no2_2011_5_mapa.png}
\captionsetup{labelformat=empty}
\caption{Mayo}
\label{fig:map-no2-2011-5}
\end{subfigure}
%
\begin{subfigure}[H]{0.15\textwidth}
\includegraphics[width=\textwidth]{C:/Users/Faraday/Desktop/practicas_alvaro/maps/media/no2/2011/no2_2011_6_mapa.png}
\captionsetup{labelformat=empty}
\caption{Junio}
\label{fig:map-no2-2011-6}
\end{subfigure}

\begin{subfigure}[H]{0.15\textwidth}
\includegraphics[width=\textwidth]{C:/Users/Faraday/Desktop/practicas_alvaro/maps/media/no2/2011/no2_2011_7_mapa.png}
\captionsetup{labelformat=empty}
\caption{Julio}
\label{fig:map-no2-2011-7}
\end{subfigure}
%
\begin{subfigure}[H]{0.15\textwidth}
\includegraphics[width=\textwidth]{C:/Users/Faraday/Desktop/practicas_alvaro/maps/media/no2/2011/no2_2011_8_mapa.png}
\captionsetup{labelformat=empty}
\caption{Agosto}
\label{fig:map-no2-2011-8}
\end{subfigure}
%
\begin{subfigure}[H]{0.15\textwidth}
\includegraphics[width=\textwidth]{C:/Users/Faraday/Desktop/practicas_alvaro/maps/media/no2/2011/no2_2011_9_mapa.png}
\captionsetup{labelformat=empty}
\caption{Septiembre}
\label{fig:map-no2-2011-9}
\end{subfigure}
%
\begin{subfigure}[H]{0.15\textwidth}
\includegraphics[width=\textwidth]{C:/Users/Faraday/Desktop/practicas_alvaro/maps/media/no2/2011/no2_2011_10_mapa.png}
\captionsetup{labelformat=empty}
\caption{Octubre}
\label{fig:map-no2-2011-10}
\end{subfigure}
%
\begin{subfigure}[H]{0.15\textwidth}
\includegraphics[width=\textwidth]{C:/Users/Faraday/Desktop/practicas_alvaro/maps/media/no2/2011/no2_2011_11_mapa.png}
\captionsetup{labelformat=empty}
\caption{Noviembre}
\label{fig:map-no2-2011-11}
\end{subfigure}
%
\begin{subfigure}[H]{0.15\textwidth}
\includegraphics[width=\textwidth]{C:/Users/Faraday/Desktop/practicas_alvaro/maps/media/no2/2011/no2_2011_12_mapa.png}
\captionsetup{labelformat=empty}
\caption{Diciembre}
\label{fig:map-no2-2011-12}
\end{subfigure}

\begin{subfigure}[H]{0.45\textwidth}
\includegraphics[width=\textwidth]{C:/Users/Faraday/Desktop/practicas_alvaro/maps/media/no2/color_scale.png}
\captionsetup{labelformat=empty}
\caption{}
\end{subfigure}

\vspace*{-7mm}
\caption{Media mensual de concentración del $NO_{2}$ en el año 2011}
\label{fig:map-no2-2011}
\end{figure}

En cuanto a la concentración anual de dióxido de nitrógeno, vemos en la figura \ref{fig:map-no2-anual} que se supera el valor límite de 40 $\nicefrac{\mu g}{m^3}$ en las ciudades y alrededores de Londres, Oxford, Brístol, Birmingham, Mánchester, Kingston Upon Hull y Sunderland. El territorio a través del cuál se extiende esta concentración aumenta o disminuye en función del año, pero se encuentra en torno al 5\%, como puede comprobarse en el cuadro \ref{tab:annual_no2}. Sin embargo, dada la insuficiente cantidad de datos anuales de los que disponemos, no es posible estimar una tendencia de evolución de la concentración de $NO_{2}$.

% We include the annual concentration maps
\begin{figure}[H]
\centering
\begin{subfigure}[H]{0.18\textwidth}
\includegraphics[width=\textwidth]{C:/Users/Faraday/Desktop/practicas_alvaro/maps/media/no2/anual/no2_2007_anual_mapa.png}
\captionsetup{labelformat=empty}
\caption{2007}
\end{subfigure}
%
\begin{subfigure}[H]{0.18\textwidth}
\includegraphics[width=\textwidth]{C:/Users/Faraday/Desktop/practicas_alvaro/maps/media/no2/anual/no2_2008_anual_mapa.png}
\captionsetup{labelformat=empty}
\caption{2008}
\end{subfigure}
%
\begin{subfigure}[H]{0.18\textwidth}
\includegraphics[width=\textwidth]{C:/Users/Faraday/Desktop/practicas_alvaro/maps/media/no2/anual/no2_2009_anual_mapa.png}
\captionsetup{labelformat=empty}
\caption{2009}
\end{subfigure}
%
\begin{subfigure}[H]{0.18\textwidth}
\includegraphics[width=\textwidth]{C:/Users/Faraday/Desktop/practicas_alvaro/maps/media/no2/anual/no2_2010_anual_mapa.png}
\captionsetup{labelformat=empty}
\caption{2010}
\end{subfigure}
%
\begin{subfigure}[H]{0.18\textwidth}
\includegraphics[width=\textwidth]{C:/Users/Faraday/Desktop/practicas_alvaro/maps/media/no2/anual/no2_2011_anual_mapa.png}
\captionsetup{labelformat=empty}
\caption{2011}
\end{subfigure}

\begin{subfigure}[H]{0.45\textwidth}
\includegraphics[width=\textwidth]{C:/Users/Faraday/Desktop/practicas_alvaro/maps/media/no2/color_scale.png}
\captionsetup{labelformat=empty}
\caption{}
\end{subfigure}

\vspace*{-7mm}
\caption{Media anual de concentración del $NO_{2}$ en el período 2007-2011}
\label{fig:map-no2-anual}
\end{figure}

% Table
\begin{table}[H]
\centering
\begin{tabularx}{\textwidth}{|c| *{2}{>{\centering\arraybackslash}X|}}
\hline
 & \multicolumn{2}{c|}{Porcentaje de la superficie de Inglaterra y Gales contaminada de acuerdo} \\
 Año & \multicolumn{2}{c|}{al rango de concentración de $NO_{2}$ ($\frac{\mu g}{m^{3}}$)} \\ \cline{2-3}
  & (0, 40) & (40, $\infty$) \\
 \hline
 2007 & 94,48 & 5,52 \\
 \hline
 2008 & 94,03 & 5,97 \\
 \hline
 2009 & 96,16 & 3,84 \\
 \hline
 2010 & 93,87 & 6,13 \\
 \hline
 2011 & 94,33 & 5,67 \\
 \hline
\end{tabularx}
\caption{}
\label{tab:annual_no2}
\end{table}

\subsection{Ozono}

Dado que el $NO_{2}$ y el $O_{3}$ están muy relacionados, sólo analizando conjuntamente los datos del dióxido de nitrógeno y del ozono, se podrá dar una explicación de la distribución de la concentración de ambos contaminantes en Inglaterra y Gales. Veamos, primero, los diagramas de caja y bigotes con la media mensual de concentración de ozono. Como puede apreciarse en la figura \ref{fig:box-ozone-monthly}, los meses de mayor concentración de ozono se producen en los meses de abril, mayo y junio, mientras que los meses de menor concentración suelen ser en los meses de otoño e invierno (sobre todo, en enero y febrero).

Con ello, se puede comprobar que tiene relación con el dióxido de nitrógeno, pues en la figura \ref{fig:box-no2-monthly} se comprobaba que la mayor concentración se produce en meses de otoño e invierno, mientras que en el caso del ozono, esa mayor concentración se produce en meses más cálidos y con mayor horas de luz solar. Del mismo modo, la menor concentración de dióxido de nitrógeno se produce en los meses de verano, mientras que la menor concentración de ozono se produce en los meses de invierno: están inversamente relacionados.

% Monthly boxplots
\begin{figure}[H]
\centering
\begin{subfigure}[H]{0.30\textwidth}
\includegraphics[width=\textwidth]{C:/Users/Faraday/Desktop/practicas_alvaro/images/boxplots/monthly/ozone_2007_boxplot_definitive.png}
\captionsetup{labelformat=empty}
\caption{2007}
\label{fig:box-ozone-2007}
\end{subfigure}
%
\begin{subfigure}[H]{0.30\textwidth}
\includegraphics[width=\textwidth]{C:/Users/Faraday/Desktop/practicas_alvaro/images/boxplots/monthly/ozone_2008_boxplot_definitive.png}
\captionsetup{labelformat=empty}
\caption{2008}
\label{fig:box-ozone-2008}
\end{subfigure}
%
\begin{subfigure}[H]{0.30\textwidth}
\includegraphics[width=\textwidth]{C:/Users/Faraday/Desktop/practicas_alvaro/images/boxplots/monthly/ozone_2009_boxplot_definitive.png}
\captionsetup{labelformat=empty}
\caption{2009}
\label{fig:box-ozone-2009}
\end{subfigure}

\begin{subfigure}[H]{0.30\textwidth}
\includegraphics[width=\textwidth]{C:/Users/Faraday/Desktop/practicas_alvaro/images/boxplots/monthly/ozone_2010_boxplot_definitive.png}
\captionsetup{labelformat=empty}
\caption{2010}
\label{fig:box-ozone-2010}
\end{subfigure}
%
\begin{subfigure}[H]{0.30\textwidth}
\includegraphics[width=\textwidth]{C:/Users/Faraday/Desktop/practicas_alvaro/images/boxplots/monthly/ozone_2011_boxplot_definitive.png}
\captionsetup{labelformat=empty}
\caption{2011}
\label{fig:box-ozone-2011}
\end{subfigure}
\caption{Diagramas de caja y bigotes de la media mensual concentración de $O_{3}$ en el período 2007-2011}
\label{fig:box-ozone-monthly}
\end{figure}

Si se analiza la distribución territorial de la concentración de ozono, se podrán sacar conclusiones de qué es lo que ocurre entre el $O_{3}$ y el $NO_{2}$.

Como puede observarse en las figuras \ref{fig:map-ozone-2007}, \ref{fig:map-ozone-2008}, \ref{fig:map-ozone-2009}, \ref{fig:map-ozone-2010} y \ref{fig:map-ozone-2011}, la concentración de ozono suele ser más alta en el campo de Inglaterra y Gales que en las ciudades, donde siempre, sea el mes que sea, la concentración de ozono es menor a la mediana del mes. Y además, muchas ciudades en las que se vio que se concentraba el $NO_{2}$ son las mismas en las que la concentración de $O_{3}$ es menor que el resto del territorio: Londres, Oxford, Birmingham, Mánchester, Kingston Upon Hill y Sunderland, entre otras.

Precisamente, la distribución de $O_{3}$ se comporta de manera contraria a la distribución de $NO_{2}$ en Inglaterra y Gales: mientras el dióxido de nitrógeno se concentra especialmente en ciudades grandes o con mucha industria (con una concentración más baja su concentración en el campo), el ozono se concentra especialmente en el campo, con las ciudades grandes como puntos en los que su concentración es mínima.

% We include the graphics. Year 2007
\begin{figure}[H]
\centering
\begin{subfigure}[H]{0.15\textwidth}
\includegraphics[width=\textwidth]{C:/Users/Faraday/Desktop/practicas_alvaro/maps/media/ozone/2007/ozone_2007_1_mapa.png}
\captionsetup{labelformat=empty}
\caption{Enero}
\label{fig:map-ozone-2007-1}
\end{subfigure}
%
\begin{subfigure}[H]{0.15\textwidth}
\includegraphics[width=\textwidth]{C:/Users/Faraday/Desktop/practicas_alvaro/maps/media/ozone/2007/ozone_2007_2_mapa.png}
\captionsetup{labelformat=empty}
\caption{Febrero}
\label{fig:map-ozone-2007-2}
\end{subfigure}
%
\begin{subfigure}[H]{0.15\textwidth}
\includegraphics[width=\textwidth]{C:/Users/Faraday/Desktop/practicas_alvaro/maps/media/ozone/2007/ozone_2007_3_mapa.png}
\captionsetup{labelformat=empty}
\caption{Marzo}
\label{fig:map-ozone-2007-3}
\end{subfigure}
%
\begin{subfigure}[H]{0.15\textwidth}
\includegraphics[width=\textwidth]{C:/Users/Faraday/Desktop/practicas_alvaro/maps/media/ozone/2007/ozone_2007_4_mapa.png}
\captionsetup{labelformat=empty}
\caption{Abril}
\label{fig:map-ozone-2007-4}
\end{subfigure}
%
\begin{subfigure}[H]{0.15\textwidth}
\includegraphics[width=\textwidth]{C:/Users/Faraday/Desktop/practicas_alvaro/maps/media/ozone/2007/ozone_2007_5_mapa.png}
\captionsetup{labelformat=empty}
\caption{Mayo}
\label{fig:map-ozone-2007-5}
\end{subfigure}
%
\begin{subfigure}[H]{0.15\textwidth}
\includegraphics[width=\textwidth]{C:/Users/Faraday/Desktop/practicas_alvaro/maps/media/ozone/2007/ozone_2007_6_mapa.png}
\captionsetup{labelformat=empty}
\caption{Junio}
\label{fig:map-ozone-2007-6}
\end{subfigure}

\begin{subfigure}[H]{0.15\textwidth}
\includegraphics[width=\textwidth]{C:/Users/Faraday/Desktop/practicas_alvaro/maps/media/ozone/2007/ozone_2007_7_mapa.png}
\captionsetup{labelformat=empty}
\caption{Julio}
\label{fig:map-ozone-2007-7}
\end{subfigure}
%
\begin{subfigure}[H]{0.15\textwidth}
\includegraphics[width=\textwidth]{C:/Users/Faraday/Desktop/practicas_alvaro/maps/media/ozone/2007/ozone_2007_8_mapa.png}
\captionsetup{labelformat=empty}
\caption{Agosto}
\label{fig:map-ozone-2007-8}
\end{subfigure}
%
\begin{subfigure}[H]{0.15\textwidth}
\includegraphics[width=\textwidth]{C:/Users/Faraday/Desktop/practicas_alvaro/maps/media/ozone/2007/ozone_2007_9_mapa.png}
\captionsetup{labelformat=empty}
\caption{Septiembre}
\label{fig:map-ozone-2007-9}
\end{subfigure}
%
\begin{subfigure}[H]{0.15\textwidth}
\includegraphics[width=\textwidth]{C:/Users/Faraday/Desktop/practicas_alvaro/maps/media/ozone/2007/ozone_2007_10_mapa.png}
\captionsetup{labelformat=empty}
\caption{Octubre}
\label{fig:map-ozone-2007-10}
\end{subfigure}
%
\begin{subfigure}[H]{0.15\textwidth}
\includegraphics[width=\textwidth]{C:/Users/Faraday/Desktop/practicas_alvaro/maps/media/ozone/2007/ozone_2007_11_mapa.png}
\captionsetup{labelformat=empty}
\caption{Noviembre}
\label{fig:map-ozone-2007-11}
\end{subfigure}
%
\begin{subfigure}[H]{0.15\textwidth}
\includegraphics[width=\textwidth]{C:/Users/Faraday/Desktop/practicas_alvaro/maps/media/ozone/2007/ozone_2007_12_mapa.png}
\captionsetup{labelformat=empty}
\caption{Diciembre}
\label{fig:map-ozone-2007-12}
\end{subfigure}

\begin{subfigure}[H]{0.45\textwidth}
\includegraphics[width=\textwidth]{C:/Users/Faraday/Desktop/practicas_alvaro/maps/media/ozone/color_scale.png}
\captionsetup{labelformat=empty}
\caption{}
\end{subfigure}

\vspace*{-7mm}
\caption{Media mensual de concentración del $O_{3}$ en el año 2007}
\label{fig:map-ozone-2007}
\end{figure}

% We include the graphics. Year 2008
\begin{figure}[H]
\centering
\begin{subfigure}[H]{0.15\textwidth}
\includegraphics[width=\textwidth]{C:/Users/Faraday/Desktop/practicas_alvaro/maps/media/ozone/2008/ozone_2008_1_mapa.png}
\captionsetup{labelformat=empty}
\caption{Enero}
\label{fig:map-ozone-2008-1}
\end{subfigure}
%
\begin{subfigure}[H]{0.15\textwidth}
\includegraphics[width=\textwidth]{C:/Users/Faraday/Desktop/practicas_alvaro/maps/media/ozone/2008/ozone_2008_2_mapa.png}
\captionsetup{labelformat=empty}
\caption{Febrero}
\label{fig:map-ozone-2008-2}
\end{subfigure}
%
\begin{subfigure}[H]{0.15\textwidth}
\includegraphics[width=\textwidth]{C:/Users/Faraday/Desktop/practicas_alvaro/maps/media/ozone/2008/ozone_2008_3_mapa.png}
\captionsetup{labelformat=empty}
\caption{Marzo}
\label{fig:map-ozone-2008-3}
\end{subfigure}
%
\begin{subfigure}[H]{0.15\textwidth}
\includegraphics[width=\textwidth]{C:/Users/Faraday/Desktop/practicas_alvaro/maps/media/ozone/2008/ozone_2008_4_mapa.png}
\captionsetup{labelformat=empty}
\caption{Abril}
\label{fig:map-ozone-2008-4}
\end{subfigure}
%
\begin{subfigure}[H]{0.15\textwidth}
\includegraphics[width=\textwidth]{C:/Users/Faraday/Desktop/practicas_alvaro/maps/media/ozone/2008/ozone_2008_5_mapa.png}
\captionsetup{labelformat=empty}
\caption{Mayo}
\label{fig:map-ozone-2008-5}
\end{subfigure}
%
\begin{subfigure}[H]{0.15\textwidth}
\includegraphics[width=\textwidth]{C:/Users/Faraday/Desktop/practicas_alvaro/maps/media/ozone/2008/ozone_2008_6_mapa.png}
\captionsetup{labelformat=empty}
\caption{Junio}
\label{fig:map-ozone-2008-6}
\end{subfigure}

\begin{subfigure}[H]{0.15\textwidth}
\includegraphics[width=\textwidth]{C:/Users/Faraday/Desktop/practicas_alvaro/maps/media/ozone/2008/ozone_2008_7_mapa.png}
\captionsetup{labelformat=empty}
\caption{Julio}
\label{fig:map-ozone-2008-7}
\end{subfigure}
%
\begin{subfigure}[H]{0.15\textwidth}
\includegraphics[width=\textwidth]{C:/Users/Faraday/Desktop/practicas_alvaro/maps/media/ozone/2008/ozone_2008_8_mapa.png}
\captionsetup{labelformat=empty}
\caption{Agosto}
\label{fig:map-ozone-2008-8}
\end{subfigure}
%
\begin{subfigure}[H]{0.15\textwidth}
\includegraphics[width=\textwidth]{C:/Users/Faraday/Desktop/practicas_alvaro/maps/media/ozone/2008/ozone_2008_9_mapa.png}
\captionsetup{labelformat=empty}
\caption{Septiembre}
\label{fig:map-ozone-2008-9}
\end{subfigure}
%
\begin{subfigure}[H]{0.15\textwidth}
\includegraphics[width=\textwidth]{C:/Users/Faraday/Desktop/practicas_alvaro/maps/media/ozone/2008/ozone_2008_10_mapa.png}
\captionsetup{labelformat=empty}
\caption{Octubre}
\label{fig:map-ozone-2008-10}
\end{subfigure}
%
\begin{subfigure}[H]{0.15\textwidth}
\includegraphics[width=\textwidth]{C:/Users/Faraday/Desktop/practicas_alvaro/maps/media/ozone/2008/ozone_2008_11_mapa.png}
\captionsetup{labelformat=empty}
\caption{Noviembre}
\label{fig:map-ozone-2008-11}
\end{subfigure}
%
\begin{subfigure}[H]{0.15\textwidth}
\includegraphics[width=\textwidth]{C:/Users/Faraday/Desktop/practicas_alvaro/maps/media/ozone/2008/ozone_2008_12_mapa.png}
\captionsetup{labelformat=empty}
\caption{Diciembre}
\label{fig:map-ozone-2008-12}
\end{subfigure}

\begin{subfigure}[H]{0.45\textwidth}
\includegraphics[width=\textwidth]{C:/Users/Faraday/Desktop/practicas_alvaro/maps/media/ozone/color_scale.png}
\captionsetup{labelformat=empty}
\caption{}
\end{subfigure}

\vspace*{-7mm}
\caption{Media mensual de concentración del $O_{3}$ en el año 2008}
\label{fig:map-ozone-2008}
\end{figure}

% We include the graphics. Year 2009
\begin{figure}[H]
\centering
\begin{subfigure}[H]{0.15\textwidth}
\includegraphics[width=\textwidth]{C:/Users/Faraday/Desktop/practicas_alvaro/maps/media/ozone/2009/ozone_2009_1_mapa.png}
\captionsetup{labelformat=empty}
\caption{Enero}
\label{fig:map-ozone-2009-1}
\end{subfigure}
%
\begin{subfigure}[H]{0.15\textwidth}
\includegraphics[width=\textwidth]{C:/Users/Faraday/Desktop/practicas_alvaro/maps/media/ozone/2009/ozone_2009_2_mapa.png}
\captionsetup{labelformat=empty}
\caption{Febrero}
\label{fig:map-ozone-2009-2}
\end{subfigure}
%
\begin{subfigure}[H]{0.15\textwidth}
\includegraphics[width=\textwidth]{C:/Users/Faraday/Desktop/practicas_alvaro/maps/media/ozone/2009/ozone_2009_3_mapa.png}
\captionsetup{labelformat=empty}
\caption{Marzo}
\label{fig:map-ozone-2009-3}
\end{subfigure}
%
\begin{subfigure}[H]{0.15\textwidth}
\includegraphics[width=\textwidth]{C:/Users/Faraday/Desktop/practicas_alvaro/maps/media/ozone/2009/ozone_2009_4_mapa.png}
\captionsetup{labelformat=empty}
\caption{Abril}
\label{fig:map-ozone-2009-4}
\end{subfigure}
%
\begin{subfigure}[H]{0.15\textwidth}
\includegraphics[width=\textwidth]{C:/Users/Faraday/Desktop/practicas_alvaro/maps/media/ozone/2009/ozone_2009_5_mapa.png}
\captionsetup{labelformat=empty}
\caption{Mayo}
\label{fig:map-ozone-2009-5}
\end{subfigure}
%
\begin{subfigure}[H]{0.15\textwidth}
\includegraphics[width=\textwidth]{C:/Users/Faraday/Desktop/practicas_alvaro/maps/media/ozone/2009/ozone_2009_6_mapa.png}
\captionsetup{labelformat=empty}
\caption{Junio}
\label{fig:map-ozone-2009-6}
\end{subfigure}

\begin{subfigure}[H]{0.15\textwidth}
\includegraphics[width=\textwidth]{C:/Users/Faraday/Desktop/practicas_alvaro/maps/media/ozone/2009/ozone_2009_7_mapa.png}
\captionsetup{labelformat=empty}
\caption{Julio}
\label{fig:map-ozone-2009-7}
\end{subfigure}
%
\begin{subfigure}[H]{0.15\textwidth}
\includegraphics[width=\textwidth]{C:/Users/Faraday/Desktop/practicas_alvaro/maps/media/ozone/2009/ozone_2009_8_mapa.png}
\captionsetup{labelformat=empty}
\caption{Agosto}
\label{fig:map-ozone-2009-8}
\end{subfigure}
%
\begin{subfigure}[H]{0.15\textwidth}
\includegraphics[width=\textwidth]{C:/Users/Faraday/Desktop/practicas_alvaro/maps/media/ozone/2009/ozone_2009_9_mapa.png}
\captionsetup{labelformat=empty}
\caption{Septiembre}
\label{fig:map-ozone-2009-9}
\end{subfigure}
%
\begin{subfigure}[H]{0.15\textwidth}
\includegraphics[width=\textwidth]{C:/Users/Faraday/Desktop/practicas_alvaro/maps/media/ozone/2009/ozone_2009_10_mapa.png}
\captionsetup{labelformat=empty}
\caption{Octubre}
\label{fig:map-ozone-2009-10}
\end{subfigure}
%
\begin{subfigure}[H]{0.15\textwidth}
\includegraphics[width=\textwidth]{C:/Users/Faraday/Desktop/practicas_alvaro/maps/media/ozone/2009/ozone_2009_11_mapa.png}
\captionsetup{labelformat=empty}
\caption{Noviembre}
\label{fig:map-ozone-2009-11}
\end{subfigure}
%
\begin{subfigure}[H]{0.15\textwidth}
\includegraphics[width=\textwidth]{C:/Users/Faraday/Desktop/practicas_alvaro/maps/media/ozone/2009/ozone_2009_12_mapa.png}
\captionsetup{labelformat=empty}
\caption{Diciembre}
\label{fig:map-ozone-2009-12}
\end{subfigure}

\begin{subfigure}[H]{0.45\textwidth}
\includegraphics[width=\textwidth]{C:/Users/Faraday/Desktop/practicas_alvaro/maps/media/ozone/color_scale.png}
\captionsetup{labelformat=empty}
\caption{}
\end{subfigure}

\vspace*{-7mm}
\caption{Media mensual de concentración del $O_{3}$ en el año 2009}
\label{fig:map-ozone-2009}
\end{figure}

% We include the graphics. Year 2010
\begin{figure}[H]
\centering
\begin{subfigure}[H]{0.15\textwidth}
\includegraphics[width=\textwidth]{C:/Users/Faraday/Desktop/practicas_alvaro/maps/media/ozone/2010/ozone_2010_1_mapa.png}
\captionsetup{labelformat=empty}
\caption{Enero}
\label{fig:map-ozone-2010-1}
\end{subfigure}
%
\begin{subfigure}[H]{0.15\textwidth}
\includegraphics[width=\textwidth]{C:/Users/Faraday/Desktop/practicas_alvaro/maps/media/ozone/2010/ozone_2010_2_mapa.png}
\captionsetup{labelformat=empty}
\caption{Febrero}
\label{fig:map-ozone-2010-2}
\end{subfigure}
%
\begin{subfigure}[H]{0.15\textwidth}
\includegraphics[width=\textwidth]{C:/Users/Faraday/Desktop/practicas_alvaro/maps/media/ozone/2010/ozone_2010_3_mapa.png}
\captionsetup{labelformat=empty}
\caption{Marzo}
\label{fig:map-ozone-2010-3}
\end{subfigure}
%
\begin{subfigure}[H]{0.15\textwidth}
\includegraphics[width=\textwidth]{C:/Users/Faraday/Desktop/practicas_alvaro/maps/media/ozone/2010/ozone_2010_4_mapa.png}
\captionsetup{labelformat=empty}
\caption{Abril}
\label{fig:map-ozone-2010-4}
\end{subfigure}
%
\begin{subfigure}[H]{0.15\textwidth}
\includegraphics[width=\textwidth]{C:/Users/Faraday/Desktop/practicas_alvaro/maps/media/ozone/2010/ozone_2010_5_mapa.png}
\captionsetup{labelformat=empty}
\caption{Mayo}
\label{fig:map-ozone-2010-5}
\end{subfigure}
%
\begin{subfigure}[H]{0.15\textwidth}
\includegraphics[width=\textwidth]{C:/Users/Faraday/Desktop/practicas_alvaro/maps/media/ozone/2010/ozone_2010_6_mapa.png}
\captionsetup{labelformat=empty}
\caption{Junio}
\label{fig:map-ozone-2010-6}
\end{subfigure}

\begin{subfigure}[H]{0.15\textwidth}
\includegraphics[width=\textwidth]{C:/Users/Faraday/Desktop/practicas_alvaro/maps/media/ozone/2010/ozone_2010_7_mapa.png}
\captionsetup{labelformat=empty}
\caption{Julio}
\label{fig:map-ozone-2010-7}
\end{subfigure}
%
\begin{subfigure}[H]{0.15\textwidth}
\includegraphics[width=\textwidth]{C:/Users/Faraday/Desktop/practicas_alvaro/maps/media/ozone/2010/ozone_2010_8_mapa.png}
\captionsetup{labelformat=empty}
\caption{Agosto}
\label{fig:map-ozone-2010-8}
\end{subfigure}
%
\begin{subfigure}[H]{0.15\textwidth}
\includegraphics[width=\textwidth]{C:/Users/Faraday/Desktop/practicas_alvaro/maps/media/ozone/2010/ozone_2010_9_mapa.png}
\captionsetup{labelformat=empty}
\caption{Septiembre}
\label{fig:map-ozone-2010-9}
\end{subfigure}
%
\begin{subfigure}[H]{0.15\textwidth}
\includegraphics[width=\textwidth]{C:/Users/Faraday/Desktop/practicas_alvaro/maps/media/ozone/2010/ozone_2010_10_mapa.png}
\captionsetup{labelformat=empty}
\caption{Octubre}
\label{fig:map-ozone-2010-10}
\end{subfigure}
%
\begin{subfigure}[H]{0.15\textwidth}
\includegraphics[width=\textwidth]{C:/Users/Faraday/Desktop/practicas_alvaro/maps/media/ozone/2010/ozone_2010_11_mapa.png}
\captionsetup{labelformat=empty}
\caption{Noviembre}
\label{fig:map-ozone-2010-11}
\end{subfigure}
%
\begin{subfigure}[H]{0.15\textwidth}
\includegraphics[width=\textwidth]{C:/Users/Faraday/Desktop/practicas_alvaro/maps/media/ozone/2010/ozone_2010_12_mapa.png}
\captionsetup{labelformat=empty}
\caption{Diciembre}
\label{fig:map-ozone-2010-12}
\end{subfigure}

\begin{subfigure}[H]{0.45\textwidth}
\includegraphics[width=\textwidth]{C:/Users/Faraday/Desktop/practicas_alvaro/maps/media/ozone/color_scale.png}
\captionsetup{labelformat=empty}
\caption{}
\end{subfigure}

\vspace*{-7mm}
\caption{Media mensual de concentración del $O_{3}$ en el año 2010}
\label{fig:map-ozone-2010}
\end{figure}

% We include the graphics. Year 2011
\begin{figure}[H]
\centering
\begin{subfigure}[H]{0.15\textwidth}
\includegraphics[width=\textwidth]{C:/Users/Faraday/Desktop/practicas_alvaro/maps/media/ozone/2011/ozone_2011_1_mapa.png}
\captionsetup{labelformat=empty}
\caption{Enero}
\label{fig:map-ozone-2011-1}
\end{subfigure}
%
\begin{subfigure}[H]{0.15\textwidth}
\includegraphics[width=\textwidth]{C:/Users/Faraday/Desktop/practicas_alvaro/maps/media/ozone/2011/ozone_2011_2_mapa.png}
\captionsetup{labelformat=empty}
\caption{Febrero}
\label{fig:map-ozone-2011-2}
\end{subfigure}
%
\begin{subfigure}[H]{0.15\textwidth}
\includegraphics[width=\textwidth]{C:/Users/Faraday/Desktop/practicas_alvaro/maps/media/ozone/2011/ozone_2011_3_mapa.png}
\captionsetup{labelformat=empty}
\caption{Marzo}
\label{fig:map-ozone-2011-3}
\end{subfigure}
%
\begin{subfigure}[H]{0.15\textwidth}
\includegraphics[width=\textwidth]{C:/Users/Faraday/Desktop/practicas_alvaro/maps/media/ozone/2011/ozone_2011_4_mapa.png}
\captionsetup{labelformat=empty}
\caption{Abril}
\label{fig:map-ozone-2011-4}
\end{subfigure}
%
\begin{subfigure}[H]{0.15\textwidth}
\includegraphics[width=\textwidth]{C:/Users/Faraday/Desktop/practicas_alvaro/maps/media/ozone/2011/ozone_2011_5_mapa.png}
\captionsetup{labelformat=empty}
\caption{Mayo}
\label{fig:map-ozone-2011-5}
\end{subfigure}
%
\begin{subfigure}[H]{0.15\textwidth}
\includegraphics[width=\textwidth]{C:/Users/Faraday/Desktop/practicas_alvaro/maps/media/ozone/2011/ozone_2011_6_mapa.png}
\captionsetup{labelformat=empty}
\caption{Junio}
\label{fig:map-ozone-2011-6}
\end{subfigure}

\begin{subfigure}[H]{0.15\textwidth}
\includegraphics[width=\textwidth]{C:/Users/Faraday/Desktop/practicas_alvaro/maps/media/ozone/2011/ozone_2011_7_mapa.png}
\captionsetup{labelformat=empty}
\caption{Julio}
\label{fig:map-ozone-2011-7}
\end{subfigure}
%
\begin{subfigure}[H]{0.15\textwidth}
\includegraphics[width=\textwidth]{C:/Users/Faraday/Desktop/practicas_alvaro/maps/media/ozone/2011/ozone_2011_8_mapa.png}
\captionsetup{labelformat=empty}
\caption{Agosto}
\label{fig:map-ozone-2011-8}
\end{subfigure}
%
\begin{subfigure}[H]{0.15\textwidth}
\includegraphics[width=\textwidth]{C:/Users/Faraday/Desktop/practicas_alvaro/maps/media/ozone/2011/ozone_2011_9_mapa.png}
\captionsetup{labelformat=empty}
\caption{Septiembre}
\label{fig:map-ozone-2011-9}
\end{subfigure}
%
\begin{subfigure}[H]{0.15\textwidth}
\includegraphics[width=\textwidth]{C:/Users/Faraday/Desktop/practicas_alvaro/maps/media/ozone/2011/ozone_2011_10_mapa.png}
\captionsetup{labelformat=empty}
\caption{Octubre}
\label{fig:map-ozone-2011-10}
\end{subfigure}
%
\begin{subfigure}[H]{0.15\textwidth}
\includegraphics[width=\textwidth]{C:/Users/Faraday/Desktop/practicas_alvaro/maps/media/ozone/2011/ozone_2011_11_mapa.png}
\captionsetup{labelformat=empty}
\caption{Noviembre}
\label{fig:map-ozone-2011-11}
\end{subfigure}
%
\begin{subfigure}[H]{0.15\textwidth}
\includegraphics[width=\textwidth]{C:/Users/Faraday/Desktop/practicas_alvaro/maps/media/ozone/2011/ozone_2011_12_mapa.png}
\captionsetup{labelformat=empty}
\caption{Diciembre}
\label{fig:map-ozone-2011-12}
\end{subfigure}

\begin{subfigure}[H]{0.45\textwidth}
\includegraphics[width=\textwidth]{C:/Users/Faraday/Desktop/practicas_alvaro/maps/media/ozone/color_scale.png}
\captionsetup{labelformat=empty}
\caption{}
\end{subfigure}

\vspace*{-7mm}
\caption{Media mensual de concentración del $O_{3}$ en el año 2011}
\label{fig:map-ozone-2011}
\end{figure}

Dada la reacciones químicas que hemos descrito en \ref{eq:no-to-no2} y \ref{eq:no2-to-no}, estas dos reacciones se producen hasta que se halla un equilibrio entre el dióxido de nitrógeno y el ozono. Así, es comprensible que, en los meses de invierno, la concentración de $O_{3}$ sea mínima y la de $NO_{2}$ máxima. Ello es debido a que hay una menor cantidad de horas de luz al día, lo que provoca que una menor cantidad de $NO_{2}$ se descomponga y, en consecuencia, se forme ozono. Sin embargo, a partir de abril, la concentración de ozono pasa a ser máxima, mientras que la del dióxido de nitrógeno pasa a ser mínima, precisamente porque hay una mayor cantidad de horas de luz al día, por lo que se descompone una mayor cantidad de dióxido de nitrógeno y, en consecuencia, se forma más ozono. Ahora bien, también hay que tener en cuenta que una de las reacciones que puede tener el $NO_{2}$ es a ácido nítrico (reacción \ref{eq:3}), por lo que no todo el $NO_{2}$ contribuye a la formación de ozono, sino también a la del ácido nítrico.

Respecto a la concentración anual de ozono, en el período estudiado no es posible observar una tendencia de la concentración a lo largo de Inglaterra y Gales, como puede observarse en la figura \ref{fig:map-ozone-anual}.

% We include the annual concentration maps
\begin{figure}[H]
\centering
\begin{subfigure}[H]{0.18\textwidth}
\includegraphics[width=\textwidth]{C:/Users/Faraday/Desktop/practicas_alvaro/maps/media/ozone/anual/ozone_2007_anual_mapa.png}
\captionsetup{labelformat=empty}
\caption{2007}
\end{subfigure}
%
\begin{subfigure}[H]{0.18\textwidth}
\includegraphics[width=\textwidth]{C:/Users/Faraday/Desktop/practicas_alvaro/maps/media/ozone/anual/ozone_2008_anual_mapa.png}
\captionsetup{labelformat=empty}
\caption{2008}
\end{subfigure}
%
\begin{subfigure}[H]{0.18\textwidth}
\includegraphics[width=\textwidth]{C:/Users/Faraday/Desktop/practicas_alvaro/maps/media/ozone/anual/ozone_2009_anual_mapa.png}
\captionsetup{labelformat=empty}
\caption{2009}
\end{subfigure}
%
\begin{subfigure}[H]{0.18\textwidth}
\includegraphics[width=\textwidth]{C:/Users/Faraday/Desktop/practicas_alvaro/maps/media/ozone/anual/ozone_2010_anual_mapa.png}
\captionsetup{labelformat=empty}
\caption{2010}
\end{subfigure}
%
\begin{subfigure}[H]{0.18\textwidth}
\includegraphics[width=\textwidth]{C:/Users/Faraday/Desktop/practicas_alvaro/maps/media/ozone/anual/ozone_2011_anual_mapa.png}
\captionsetup{labelformat=empty}
\caption{2011}
\end{subfigure}

\begin{subfigure}[H]{0.45\textwidth}
\includegraphics[width=\textwidth]{C:/Users/Faraday/Desktop/practicas_alvaro/maps/media/ozone/color_scale.png}
\captionsetup{labelformat=empty}
\caption{}
\end{subfigure}

\vspace*{-7mm}
\caption{Media anual de concentración del $O_{3}$ en el período 2007-2011}
\label{fig:map-ozone-anual}
\end{figure}

\subsection*{Partículas \textbf{\texorpdfstring{$PM_{10}$}{PM10}}}
\addcontentsline{toc}{subsection}{\protect\numberline{4.3.}Partículas $PM_{10}$}%

Como puede verse en la figura \ref{fig:box-pm10-monthly}, cada año es muy disparejo respecto a la amplitud de la concentración. Por ejemplo, en el año 2009, las concentraciones de cada mes están muy agrupadas en ciertos valores, mientras que la concentración de cada mes del año 2011 son muy dispersas. Ahora bien, a pesar de las diferencias en concentración año a año, la distribución a lo largo de un año dado parece seguir un patrón. Por ejemplo, en los meses de verano, la concentración de $PM_{10}$ es mínima, mientras que en los meses de invierno y primavera es máxima.

% Monthly boxplots
\begin{figure}[H]
\centering
\begin{subfigure}[H]{0.30\textwidth}
\includegraphics[width=\textwidth]{C:/Users/Faraday/Desktop/practicas_alvaro/images/boxplots/monthly/pm10_2007_boxplot_definitive.png}
\captionsetup{labelformat=empty}
\caption{2007}
\label{fig:box-pm10-2007}
\end{subfigure}
%
\begin{subfigure}[H]{0.30\textwidth}
\includegraphics[width=\textwidth]{C:/Users/Faraday/Desktop/practicas_alvaro/images/boxplots/monthly/pm10_2008_boxplot_definitive.png}
\captionsetup{labelformat=empty}
\caption{2008}
\label{fig:box-pm10-2008}
\end{subfigure}
%
\begin{subfigure}[H]{0.30\textwidth}
\includegraphics[width=\textwidth]{C:/Users/Faraday/Desktop/practicas_alvaro/images/boxplots/monthly/pm10_2009_boxplot_definitive.png}
\captionsetup{labelformat=empty}
\caption{2009}
\label{fig:box-pm10-2009}
\end{subfigure}

\begin{subfigure}[H]{0.30\textwidth}
\includegraphics[width=\textwidth]{C:/Users/Faraday/Desktop/practicas_alvaro/images/boxplots/monthly/pm10_2010_boxplot_definitive.png}
\captionsetup{labelformat=empty}
\caption{2010}
\label{fig:box-pm10-2010}
\end{subfigure}
%
\begin{subfigure}[H]{0.30\textwidth}
\includegraphics[width=\textwidth]{C:/Users/Faraday/Desktop/practicas_alvaro/images/boxplots/monthly/pm10_2011_boxplot_definitive.png}
\captionsetup{labelformat=empty}
\caption{2011}
\label{fig:box-pm10-2011}
\end{subfigure}
\caption{Diagramas de caja y bigotes de la media mensual concentración de $PM_{10}$ en el período 2007-2011}
\label{fig:box-pm10-monthly}
\end{figure}

Analizando las figuras \ref{fig:map-pm10-2007}, \ref{fig:map-pm10-2008}, \ref{fig:map-pm10-2009}, \ref{fig:map-pm10-2010} y \ref{fig:map-pm10-2011}, puede observarse que la distribución de $PM_{10}$ suele ser bastante homogénea. Sin embargo, hay una tendencia en que la concentración de este contaminante sea mayor al este de Inglaterra que en Gales. Ésto sea debido a, probablemente, la lluvia que cae en cada territorio: el oeste de Inglaterra es mucho más seco que Gales, lo cual favorece que se acumule más $PM_{10}$.

% We include the graphics. Year 2007
\begin{figure}[H]
\centering
\begin{subfigure}[H]{0.15\textwidth}
\includegraphics[width=\textwidth]{C:/Users/Faraday/Desktop/practicas_alvaro/maps/media/pm10/2007/pm10_2007_1_mapa.png}
\captionsetup{labelformat=empty}
\caption{Enero}
\label{fig:map-pm10-2007-1}
\end{subfigure}
%
\begin{subfigure}[H]{0.15\textwidth}
\includegraphics[width=\textwidth]{C:/Users/Faraday/Desktop/practicas_alvaro/maps/media/pm10/2007/pm10_2007_2_mapa.png}
\captionsetup{labelformat=empty}
\caption{Febrero}
\label{fig:map-pm10-2007-2}
\end{subfigure}
%
\begin{subfigure}[H]{0.15\textwidth}
\includegraphics[width=\textwidth]{C:/Users/Faraday/Desktop/practicas_alvaro/maps/media/pm10/2007/pm10_2007_3_mapa.png}
\captionsetup{labelformat=empty}
\caption{Marzo}
\label{fig:map-pm10-2007-3}
\end{subfigure}
%
\begin{subfigure}[H]{0.15\textwidth}
\includegraphics[width=\textwidth]{C:/Users/Faraday/Desktop/practicas_alvaro/maps/media/pm10/2007/pm10_2007_4_mapa.png}
\captionsetup{labelformat=empty}
\caption{Abril}
\label{fig:map-pm10-2007-4}
\end{subfigure}
%
\begin{subfigure}[H]{0.15\textwidth}
\includegraphics[width=\textwidth]{C:/Users/Faraday/Desktop/practicas_alvaro/maps/media/pm10/2007/pm10_2007_5_mapa.png}
\captionsetup{labelformat=empty}
\caption{Mayo}
\label{fig:map-pm10-2007-5}
\end{subfigure}
%
\begin{subfigure}[H]{0.15\textwidth}
\includegraphics[width=\textwidth]{C:/Users/Faraday/Desktop/practicas_alvaro/maps/media/pm10/2007/pm10_2007_6_mapa.png}
\captionsetup{labelformat=empty}
\caption{Junio}
\label{fig:map-pm10-2007-6}
\end{subfigure}

\begin{subfigure}[H]{0.15\textwidth}
\includegraphics[width=\textwidth]{C:/Users/Faraday/Desktop/practicas_alvaro/maps/media/pm10/2007/pm10_2007_7_mapa.png}
\captionsetup{labelformat=empty}
\caption{Julio}
\label{fig:map-pm10-2007-7}
\end{subfigure}
%
\begin{subfigure}[H]{0.15\textwidth}
\includegraphics[width=\textwidth]{C:/Users/Faraday/Desktop/practicas_alvaro/maps/media/pm10/2007/pm10_2007_8_mapa.png}
\captionsetup{labelformat=empty}
\caption{Agosto}
\label{fig:map-pm10-2007-8}
\end{subfigure}
%
\begin{subfigure}[H]{0.15\textwidth}
\includegraphics[width=\textwidth]{C:/Users/Faraday/Desktop/practicas_alvaro/maps/media/pm10/2007/pm10_2007_9_mapa.png}
\captionsetup{labelformat=empty}
\caption{Septiembre}
\label{fig:map-pm10-2007-9}
\end{subfigure}
%
\begin{subfigure}[H]{0.15\textwidth}
\includegraphics[width=\textwidth]{C:/Users/Faraday/Desktop/practicas_alvaro/maps/media/pm10/2007/pm10_2007_10_mapa.png}
\captionsetup{labelformat=empty}
\caption{Octubre}
\label{fig:map-pm10-2007-10}
\end{subfigure}
%
\begin{subfigure}[H]{0.15\textwidth}
\includegraphics[width=\textwidth]{C:/Users/Faraday/Desktop/practicas_alvaro/maps/media/pm10/2007/pm10_2007_11_mapa.png}
\captionsetup{labelformat=empty}
\caption{Noviembre}
\label{fig:map-pm10-2007-11}
\end{subfigure}
%
\begin{subfigure}[H]{0.15\textwidth}
\includegraphics[width=\textwidth]{C:/Users/Faraday/Desktop/practicas_alvaro/maps/media/pm10/2007/pm10_2007_12_mapa.png}
\captionsetup{labelformat=empty}
\caption{Diciembre}
\label{fig:map-pm10-2007-12}
\end{subfigure}

\begin{subfigure}[H]{0.45\textwidth}
\includegraphics[width=\textwidth]{C:/Users/Faraday/Desktop/practicas_alvaro/maps/media/pm10/color_scale.png}
\captionsetup{labelformat=empty}
\caption{}
\end{subfigure}

\vspace*{-7mm}
\caption{Media mensual de concentración del $PM_{10}$ en el año 2007}
\label{fig:map-pm10-2007}
\end{figure}

% We include the graphics. Year 2008
\begin{figure}[H]
\centering
\begin{subfigure}[H]{0.15\textwidth}
\includegraphics[width=\textwidth]{C:/Users/Faraday/Desktop/practicas_alvaro/maps/media/pm10/2008/pm10_2008_1_mapa.png}
\captionsetup{labelformat=empty}
\caption{Enero}
\label{fig:map-pm10-2008-1}
\end{subfigure}
%
\begin{subfigure}[H]{0.15\textwidth}
\includegraphics[width=\textwidth]{C:/Users/Faraday/Desktop/practicas_alvaro/maps/media/pm10/2008/pm10_2008_2_mapa.png}
\captionsetup{labelformat=empty}
\caption{Febrero}
\label{fig:map-pm10-2008-2}
\end{subfigure}
%
\begin{subfigure}[H]{0.15\textwidth}
\includegraphics[width=\textwidth]{C:/Users/Faraday/Desktop/practicas_alvaro/maps/media/pm10/2008/pm10_2008_3_mapa.png}
\captionsetup{labelformat=empty}
\caption{Marzo}
\label{fig:map-pm10-2008-3}
\end{subfigure}
%
\begin{subfigure}[H]{0.15\textwidth}
\includegraphics[width=\textwidth]{C:/Users/Faraday/Desktop/practicas_alvaro/maps/media/pm10/2008/pm10_2008_4_mapa.png}
\captionsetup{labelformat=empty}
\caption{Abril}
\label{fig:map-pm10-2008-4}
\end{subfigure}
%
\begin{subfigure}[H]{0.15\textwidth}
\includegraphics[width=\textwidth]{C:/Users/Faraday/Desktop/practicas_alvaro/maps/media/pm10/2008/pm10_2008_5_mapa.png}
\captionsetup{labelformat=empty}
\caption{Mayo}
\label{fig:map-pm10-2008-5}
\end{subfigure}
%
\begin{subfigure}[H]{0.15\textwidth}
\includegraphics[width=\textwidth]{C:/Users/Faraday/Desktop/practicas_alvaro/maps/media/pm10/2008/pm10_2008_6_mapa.png}
\captionsetup{labelformat=empty}
\caption{Junio}
\label{fig:map-pm10-2008-6}
\end{subfigure}

\begin{subfigure}[H]{0.15\textwidth}
\includegraphics[width=\textwidth]{C:/Users/Faraday/Desktop/practicas_alvaro/maps/media/pm10/2008/pm10_2008_7_mapa.png}
\captionsetup{labelformat=empty}
\caption{Julio}
\label{fig:map-pm10-2008-7}
\end{subfigure}
%
\begin{subfigure}[H]{0.15\textwidth}
\includegraphics[width=\textwidth]{C:/Users/Faraday/Desktop/practicas_alvaro/maps/media/pm10/2008/pm10_2008_8_mapa.png}
\captionsetup{labelformat=empty}
\caption{Agosto}
\label{fig:map-pm10-2008-8}
\end{subfigure}
%
\begin{subfigure}[H]{0.15\textwidth}
\includegraphics[width=\textwidth]{C:/Users/Faraday/Desktop/practicas_alvaro/maps/media/pm10/2008/pm10_2008_9_mapa.png}
\captionsetup{labelformat=empty}
\caption{Septiembre}
\label{fig:map-pm10-2008-9}
\end{subfigure}
%
\begin{subfigure}[H]{0.15\textwidth}
\includegraphics[width=\textwidth]{C:/Users/Faraday/Desktop/practicas_alvaro/maps/media/pm10/2008/pm10_2008_10_mapa.png}
\captionsetup{labelformat=empty}
\caption{Octubre}
\label{fig:map-pm10-2008-10}
\end{subfigure}
%
\begin{subfigure}[H]{0.15\textwidth}
\includegraphics[width=\textwidth]{C:/Users/Faraday/Desktop/practicas_alvaro/maps/media/pm10/2008/pm10_2008_11_mapa.png}
\captionsetup{labelformat=empty}
\caption{Noviembre}
\label{fig:map-pm10-2008-11}
\end{subfigure}
%
\begin{subfigure}[H]{0.15\textwidth}
\includegraphics[width=\textwidth]{C:/Users/Faraday/Desktop/practicas_alvaro/maps/media/pm10/2008/pm10_2008_12_mapa.png}
\captionsetup{labelformat=empty}
\caption{Diciembre}
\label{fig:map-pm10-2008-12}
\end{subfigure}

\begin{subfigure}[H]{0.45\textwidth}
\includegraphics[width=\textwidth]{C:/Users/Faraday/Desktop/practicas_alvaro/maps/media/pm10/color_scale.png}
\captionsetup{labelformat=empty}
\caption{}
\end{subfigure}

\vspace*{-7mm}
\caption{Media mensual de concentración del $PM_{10}$ en el año 2008}
\label{fig:map-pm10-2008}
\end{figure}

% We include the graphics. Year 2009
\begin{figure}[H]
\centering
\begin{subfigure}[H]{0.15\textwidth}
\includegraphics[width=\textwidth]{C:/Users/Faraday/Desktop/practicas_alvaro/maps/media/pm10/2009/pm10_2009_1_mapa.png}
\captionsetup{labelformat=empty}
\caption{Enero}
\label{fig:map-pm10-2009-1}
\end{subfigure}
%
\begin{subfigure}[H]{0.15\textwidth}
\includegraphics[width=\textwidth]{C:/Users/Faraday/Desktop/practicas_alvaro/maps/media/pm10/2009/pm10_2009_2_mapa.png}
\captionsetup{labelformat=empty}
\caption{Febrero}
\label{fig:map-pm10-2009-2}
\end{subfigure}
%
\begin{subfigure}[H]{0.15\textwidth}
\includegraphics[width=\textwidth]{C:/Users/Faraday/Desktop/practicas_alvaro/maps/media/pm10/2009/pm10_2009_3_mapa.png}
\captionsetup{labelformat=empty}
\caption{Marzo}
\label{fig:map-pm10-2009-3}
\end{subfigure}
%
\begin{subfigure}[H]{0.15\textwidth}
\includegraphics[width=\textwidth]{C:/Users/Faraday/Desktop/practicas_alvaro/maps/media/pm10/2009/pm10_2009_4_mapa.png}
\captionsetup{labelformat=empty}
\caption{Abril}
\label{fig:map-pm10-2009-4}
\end{subfigure}
%
\begin{subfigure}[H]{0.15\textwidth}
\includegraphics[width=\textwidth]{C:/Users/Faraday/Desktop/practicas_alvaro/maps/media/pm10/2009/pm10_2009_5_mapa.png}
\captionsetup{labelformat=empty}
\caption{Mayo}
\label{fig:map-pm10-2009-5}
\end{subfigure}
%
\begin{subfigure}[H]{0.15\textwidth}
\includegraphics[width=\textwidth]{C:/Users/Faraday/Desktop/practicas_alvaro/maps/media/pm10/2009/pm10_2009_6_mapa.png}
\captionsetup{labelformat=empty}
\caption{Junio}
\label{fig:map-pm10-2009-6}
\end{subfigure}

\begin{subfigure}[H]{0.15\textwidth}
\includegraphics[width=\textwidth]{C:/Users/Faraday/Desktop/practicas_alvaro/maps/media/pm10/2009/pm10_2009_7_mapa.png}
\captionsetup{labelformat=empty}
\caption{Julio}
\label{fig:map-pm10-2009-7}
\end{subfigure}
%
\begin{subfigure}[H]{0.15\textwidth}
\includegraphics[width=\textwidth]{C:/Users/Faraday/Desktop/practicas_alvaro/maps/media/pm10/2009/pm10_2009_8_mapa.png}
\captionsetup{labelformat=empty}
\caption{Agosto}
\label{fig:map-pm10-2009-8}
\end{subfigure}
%
\begin{subfigure}[H]{0.15\textwidth}
\includegraphics[width=\textwidth]{C:/Users/Faraday/Desktop/practicas_alvaro/maps/media/pm10/2009/pm10_2009_9_mapa.png}
\captionsetup{labelformat=empty}
\caption{Septiembre}
\label{fig:map-pm10-2009-9}
\end{subfigure}
%
\begin{subfigure}[H]{0.15\textwidth}
\includegraphics[width=\textwidth]{C:/Users/Faraday/Desktop/practicas_alvaro/maps/media/pm10/2009/pm10_2009_10_mapa.png}
\captionsetup{labelformat=empty}
\caption{Octubre}
\label{fig:map-pm10-2009-10}
\end{subfigure}
%
\begin{subfigure}[H]{0.15\textwidth}
\includegraphics[width=\textwidth]{C:/Users/Faraday/Desktop/practicas_alvaro/maps/media/pm10/2009/pm10_2009_11_mapa.png}
\captionsetup{labelformat=empty}
\caption{Noviembre}
\label{fig:map-pm10-2009-11}
\end{subfigure}
%
\begin{subfigure}[H]{0.15\textwidth}
\includegraphics[width=\textwidth]{C:/Users/Faraday/Desktop/practicas_alvaro/maps/media/pm10/2009/pm10_2009_12_mapa.png}
\captionsetup{labelformat=empty}
\caption{Diciembre}
\label{fig:map-pm10-2009-12}
\end{subfigure}

\begin{subfigure}[H]{0.45\textwidth}
\includegraphics[width=\textwidth]{C:/Users/Faraday/Desktop/practicas_alvaro/maps/media/pm10/color_scale.png}
\captionsetup{labelformat=empty}
\caption{}
\end{subfigure}

\vspace*{-7mm}
\caption{Media mensual de concentración del $PM_{10}$ en el año 2009}
\label{fig:map-pm10-2009}
\end{figure}

% We include the graphics. Year 2010
\begin{figure}[H]
\centering
\begin{subfigure}[H]{0.15\textwidth}
\includegraphics[width=\textwidth]{C:/Users/Faraday/Desktop/practicas_alvaro/maps/media/pm10/2010/pm10_2010_1_mapa.png}
\captionsetup{labelformat=empty}
\caption{Enero}
\label{fig:map-pm10-2010-1}
\end{subfigure}
%
\begin{subfigure}[H]{0.15\textwidth}
\includegraphics[width=\textwidth]{C:/Users/Faraday/Desktop/practicas_alvaro/maps/media/pm10/2010/pm10_2010_2_mapa.png}
\captionsetup{labelformat=empty}
\caption{Febrero}
\label{fig:map-pm10-2010-2}
\end{subfigure}
%
\begin{subfigure}[H]{0.15\textwidth}
\includegraphics[width=\textwidth]{C:/Users/Faraday/Desktop/practicas_alvaro/maps/media/pm10/2010/pm10_2010_3_mapa.png}
\captionsetup{labelformat=empty}
\caption{Marzo}
\label{fig:map-pm10-2010-3}
\end{subfigure}
%
\begin{subfigure}[H]{0.15\textwidth}
\includegraphics[width=\textwidth]{C:/Users/Faraday/Desktop/practicas_alvaro/maps/media/pm10/2010/pm10_2010_4_mapa.png}
\captionsetup{labelformat=empty}
\caption{Abril}
\label{fig:map-pm10-2010-4}
\end{subfigure}
%
\begin{subfigure}[H]{0.15\textwidth}
\includegraphics[width=\textwidth]{C:/Users/Faraday/Desktop/practicas_alvaro/maps/media/pm10/2010/pm10_2010_5_mapa.png}
\captionsetup{labelformat=empty}
\caption{Mayo}
\label{fig:map-pm10-2010-5}
\end{subfigure}
%
\begin{subfigure}[H]{0.15\textwidth}
\includegraphics[width=\textwidth]{C:/Users/Faraday/Desktop/practicas_alvaro/maps/media/pm10/2010/pm10_2010_6_mapa.png}
\captionsetup{labelformat=empty}
\caption{Junio}
\label{fig:map-pm10-2010-6}
\end{subfigure}

\begin{subfigure}[H]{0.15\textwidth}
\includegraphics[width=\textwidth]{C:/Users/Faraday/Desktop/practicas_alvaro/maps/media/pm10/2010/pm10_2010_7_mapa.png}
\captionsetup{labelformat=empty}
\caption{Julio}
\label{fig:map-pm10-2010-7}
\end{subfigure}
%
\begin{subfigure}[H]{0.15\textwidth}
\includegraphics[width=\textwidth]{C:/Users/Faraday/Desktop/practicas_alvaro/maps/media/pm10/2010/pm10_2010_8_mapa.png}
\captionsetup{labelformat=empty}
\caption{Agosto}
\label{fig:map-pm10-2010-8}
\end{subfigure}
%
\begin{subfigure}[H]{0.15\textwidth}
\includegraphics[width=\textwidth]{C:/Users/Faraday/Desktop/practicas_alvaro/maps/media/pm10/2010/pm10_2010_9_mapa.png}
\captionsetup{labelformat=empty}
\caption{Septiembre}
\label{fig:map-pm10-2010-9}
\end{subfigure}
%
\begin{subfigure}[H]{0.15\textwidth}
\includegraphics[width=\textwidth]{C:/Users/Faraday/Desktop/practicas_alvaro/maps/media/pm10/2010/pm10_2010_10_mapa.png}
\captionsetup{labelformat=empty}
\caption{Octubre}
\label{fig:map-pm10-2010-10}
\end{subfigure}
%
\begin{subfigure}[H]{0.15\textwidth}
\includegraphics[width=\textwidth]{C:/Users/Faraday/Desktop/practicas_alvaro/maps/media/pm10/2010/pm10_2010_11_mapa.png}
\captionsetup{labelformat=empty}
\caption{Noviembre}
\label{fig:map-pm10-2010-11}
\end{subfigure}
%
\begin{subfigure}[H]{0.15\textwidth}
\includegraphics[width=\textwidth]{C:/Users/Faraday/Desktop/practicas_alvaro/maps/media/pm10/2010/pm10_2010_12_mapa.png}
\captionsetup{labelformat=empty}
\caption{Diciembre}
\label{fig:map-pm10-2010-12}
\end{subfigure}

\begin{subfigure}[H]{0.45\textwidth}
\includegraphics[width=\textwidth]{C:/Users/Faraday/Desktop/practicas_alvaro/maps/media/pm10/color_scale.png}
\captionsetup{labelformat=empty}
\caption{}
\end{subfigure}

\vspace*{-7mm}
\caption{Media mensual de concentración del $PM_{10}$ en el año 2010}
\label{fig:map-pm10-2010}
\end{figure}

% We include the graphics. Year 2011
\begin{figure}[H]
\centering
\begin{subfigure}[H]{0.15\textwidth}
\includegraphics[width=\textwidth]{C:/Users/Faraday/Desktop/practicas_alvaro/maps/media/pm10/2011/pm10_2011_1_mapa.png}
\captionsetup{labelformat=empty}
\caption{Enero}
\label{fig:map-pm10-2011-1}
\end{subfigure}
%
\begin{subfigure}[H]{0.15\textwidth}
\includegraphics[width=\textwidth]{C:/Users/Faraday/Desktop/practicas_alvaro/maps/media/pm10/2011/pm10_2011_2_mapa.png}
\captionsetup{labelformat=empty}
\caption{Febrero}
\label{fig:map-pm10-2011-2}
\end{subfigure}
%
\begin{subfigure}[H]{0.15\textwidth}
\includegraphics[width=\textwidth]{C:/Users/Faraday/Desktop/practicas_alvaro/maps/media/pm10/2011/pm10_2011_3_mapa.png}
\captionsetup{labelformat=empty}
\caption{Marzo}
\label{fig:map-pm10-2011-3}
\end{subfigure}
%
\begin{subfigure}[H]{0.15\textwidth}
\includegraphics[width=\textwidth]{C:/Users/Faraday/Desktop/practicas_alvaro/maps/media/pm10/2011/pm10_2011_4_mapa.png}
\captionsetup{labelformat=empty}
\caption{Abril}
\label{fig:map-pm10-2011-4}
\end{subfigure}
%
\begin{subfigure}[H]{0.15\textwidth}
\includegraphics[width=\textwidth]{C:/Users/Faraday/Desktop/practicas_alvaro/maps/media/pm10/2011/pm10_2011_5_mapa.png}
\captionsetup{labelformat=empty}
\caption{Mayo}
\label{fig:map-pm10-2011-5}
\end{subfigure}
%
\begin{subfigure}[H]{0.15\textwidth}
\includegraphics[width=\textwidth]{C:/Users/Faraday/Desktop/practicas_alvaro/maps/media/pm10/2011/pm10_2011_6_mapa.png}
\captionsetup{labelformat=empty}
\caption{Junio}
\label{fig:map-pm10-2011-6}
\end{subfigure}

\begin{subfigure}[H]{0.15\textwidth}
\includegraphics[width=\textwidth]{C:/Users/Faraday/Desktop/practicas_alvaro/maps/media/pm10/2011/pm10_2011_7_mapa.png}
\captionsetup{labelformat=empty}
\caption{Julio}
\label{fig:map-pm10-2011-7}
\end{subfigure}
%
\begin{subfigure}[H]{0.15\textwidth}
\includegraphics[width=\textwidth]{C:/Users/Faraday/Desktop/practicas_alvaro/maps/media/pm10/2011/pm10_2011_8_mapa.png}
\captionsetup{labelformat=empty}
\caption{Agosto}
\label{fig:map-pm10-2011-8}
\end{subfigure}
%
\begin{subfigure}[H]{0.15\textwidth}
\includegraphics[width=\textwidth]{C:/Users/Faraday/Desktop/practicas_alvaro/maps/media/pm10/2011/pm10_2011_9_mapa.png}
\captionsetup{labelformat=empty}
\caption{Septiembre}
\label{fig:map-pm10-2011-9}
\end{subfigure}
%
\begin{subfigure}[H]{0.15\textwidth}
\includegraphics[width=\textwidth]{C:/Users/Faraday/Desktop/practicas_alvaro/maps/media/pm10/2011/pm10_2011_10_mapa.png}
\captionsetup{labelformat=empty}
\caption{Octubre}
\label{fig:map-pm10-2011-10}
\end{subfigure}
%
\begin{subfigure}[H]{0.15\textwidth}
\includegraphics[width=\textwidth]{C:/Users/Faraday/Desktop/practicas_alvaro/maps/media/pm10/2011/pm10_2011_11_mapa.png}
\captionsetup{labelformat=empty}
\caption{Noviembre}
\label{fig:map-pm10-2011-11}
\end{subfigure}
%
\begin{subfigure}[H]{0.15\textwidth}
\includegraphics[width=\textwidth]{C:/Users/Faraday/Desktop/practicas_alvaro/maps/media/pm10/2011/pm10_2011_12_mapa.png}
\captionsetup{labelformat=empty}
\caption{Diciembre}
\label{fig:map-pm10-2011-12}
\end{subfigure}

\begin{subfigure}[H]{0.45\textwidth}
\includegraphics[width=\textwidth]{C:/Users/Faraday/Desktop/practicas_alvaro/maps/media/pm10/color_scale.png}
\captionsetup{labelformat=empty}
\caption{}
\end{subfigure}

\vspace*{-7mm}
\caption{Media mensual de concentración del $PM_{10}$ en el año 2011}
\label{fig:map-pm10-2011}
\end{figure}

Si se analiza la distribución de la concentración anual de $PM_{10}$ a lo largo del territorio de Inglaterra y Gales en el período 2007-2011 (figura \ref{fig:map-pm10-anual}), se observa que sigue una distribución parecida a la analizada anteriormente: en el oeste de Gales hay una menor concentración, mientras que en el centro y este de Inglaterra se concentra más. Respecto al porcentaje de territorio que cumple la normativa, se puede decir que en relación al $PM_{10}$, el territorio de Inglaterra y Gales está poco contaminado: en todo el periodo del 2007-2011, ningún porcentaje del territorio ha superado el límite establecido por la UE. Ahora bien, respecto al límite establecido por la OMS, en la mayoría de los años no ha sido superado por una fracción significativa del territorio, salvo en los años 2007 (un 92\% del territorio) y 2011 (un 58\% del territorio).

% We include the annual concentration maps
\begin{figure}[H]
\centering
\begin{subfigure}[H]{0.18\textwidth}
\includegraphics[width=\textwidth]{C:/Users/Faraday/Desktop/practicas_alvaro/maps/media/pm10/anual/pm10_2007_anual_mapa.png}
\captionsetup{labelformat=empty}
\caption{2007}
\end{subfigure}
%
\begin{subfigure}[H]{0.18\textwidth}
\includegraphics[width=\textwidth]{C:/Users/Faraday/Desktop/practicas_alvaro/maps/media/pm10/anual/pm10_2008_anual_mapa.png}
\captionsetup{labelformat=empty}
\caption{2008}
\end{subfigure}
%
\begin{subfigure}[H]{0.18\textwidth}
\includegraphics[width=\textwidth]{C:/Users/Faraday/Desktop/practicas_alvaro/maps/media/pm10/anual/pm10_2009_anual_mapa.png}
\captionsetup{labelformat=empty}
\caption{2009}
\end{subfigure}
%
\begin{subfigure}[H]{0.18\textwidth}
\includegraphics[width=\textwidth]{C:/Users/Faraday/Desktop/practicas_alvaro/maps/media/pm10/anual/pm10_2010_anual_mapa.png}
\captionsetup{labelformat=empty}
\caption{2010}
\end{subfigure}
%
\begin{subfigure}[H]{0.18\textwidth}
\includegraphics[width=\textwidth]{C:/Users/Faraday/Desktop/practicas_alvaro/maps/media/pm10/anual/pm10_2011_anual_mapa.png}
\captionsetup{labelformat=empty}
\caption{2011}
\end{subfigure}

\begin{subfigure}[H]{0.45\textwidth}
\includegraphics[width=\textwidth]{C:/Users/Faraday/Desktop/practicas_alvaro/maps/media/pm10/color_scale.png}
\captionsetup{labelformat=empty}
\caption{}
\end{subfigure}

\vspace*{-7mm}
\caption{Media anual de concentración del $PM_{10}$ en el período 2007-2011}
\label{fig:map-pm10-anual}
\end{figure}

% Table
\begin{table}[H]
\centering
\begin{tabularx}{\textwidth}{|c| *{3}{>{\centering\arraybackslash}X|}}
\hline
 & \multicolumn{3}{c|}{Porcentaje de la superficie de Inglaterra y Gales contaminada de acuerdo} \\
 Año & \multicolumn{3}{c|}{al rango de concentración de $PM_{10}$ ($\frac{\mu g}{m^{3}}$)} \\ \cline{2-4}
  & (0, 20) & (20, 40) & (40, $\infty$) \\
 \hline
 2007 & 8,16 & 91,84 & 0,00 \\
 \hline
 2008 & 99,96 & 0,04 & 0,00 \\
 \hline
 2009 & 99,99 & 0,01 & 0,00 \\
 \hline
 2010 & 99,89 & 0,11 & 0,00 \\
 \hline
 2011 & 42,33 & 57,67 & 0,00 \\
 \hline
\end{tabularx}
\label{table:annual_pm10}
\caption{}
\end{table}

\subsection*{Partículas \textbf{\texorpdfstring{$PM_{2,5}$}{PM2,5}}}
\addcontentsline{toc}{subsection}{\protect\numberline{4.4.}Partículas $PM_{2,5}$}%

Como puede observarse en la figura \ref{fig:box-pm2p5-monthly}, el $PM_{2,5}$ varía mucho según el año: en algunos está muy concentrado en ciertos valores (como en el 2009), mientras que en otros el rango de concentración de $PM_{2,5}$ es muy amplio (como en el 2008). Sin embargo, en todos puede verse una tendencia de haber menor concentración en los meses de verano, mientras que la concentración es más alta a principios de año. Es posible que esta tendencia esté relacionada con la mayor concentración de $NO_{2}$ en los meses de invierno y primavera (pues los nitratos son precursores del $PM_{2,5}$); sin embargo, dada la amplia gama de precursores de este contaminante, hacen falta más datos para poder afirmar ésto.

% Monthly boxplots
\begin{figure}[H]
\centering
\begin{subfigure}[H]{0.30\textwidth}
\includegraphics[width=\textwidth]{C:/Users/Faraday/Desktop/practicas_alvaro/images/boxplots/monthly/pm2p5_2007_boxplot_definitive.png}
\captionsetup{labelformat=empty}
\caption{2007}
\label{fig:box-pm2p5-2007}
\end{subfigure}
%
\begin{subfigure}[H]{0.30\textwidth}
\includegraphics[width=\textwidth]{C:/Users/Faraday/Desktop/practicas_alvaro/images/boxplots/monthly/pm2p5_2008_boxplot_definitive.png}
\captionsetup{labelformat=empty}
\caption{2008}
\label{fig:box-pm2p5-2008}
\end{subfigure}
%
\begin{subfigure}[H]{0.30\textwidth}
\includegraphics[width=\textwidth]{C:/Users/Faraday/Desktop/practicas_alvaro/images/boxplots/monthly/pm2p5_2009_boxplot_definitive.png}
\captionsetup{labelformat=empty}
\caption{2009}
\label{fig:box-pm2p5-2009}
\end{subfigure}

\begin{subfigure}[H]{0.30\textwidth}
\includegraphics[width=\textwidth]{C:/Users/Faraday/Desktop/practicas_alvaro/images/boxplots/monthly/pm2p5_2010_boxplot_definitive.png}
\captionsetup{labelformat=empty}
\caption{2010}
\label{fig:box-pm2p5-2010}
\end{subfigure}
%
\begin{subfigure}[H]{0.30\textwidth}
\includegraphics[width=\textwidth]{C:/Users/Faraday/Desktop/practicas_alvaro/images/boxplots/monthly/pm2p5_2011_boxplot_definitive.png}
\captionsetup{labelformat=empty}
\caption{2011}
\label{fig:box-pm2p5-2011}
\end{subfigure}
\caption{Diagramas de caja y bigotes de la media mensual concentración de $PM_{2,5}$ en el período 2007-2011}
\label{fig:box-pm2p5-monthly}
\end{figure}

Si se observan las figuras \ref{fig:map-pm2p5-2007}, \ref{fig:map-pm2p5-2008}, \ref{fig:map-pm2p5-2009}, \ref{fig:map-pm2p5-2010} y \ref{fig:map-pm2p5-2011}, se puede comprobar que la distribución de $PM_{2,5}$ a lo largo de Inglaterra y Gales es distinta año a año, no parece seguir ningún patrón de distribución. Lo único que puede decirse es que parece que en el territorio de Gales tiende a concentrarse menos $PM_{2,5}$ que en Inglaterra, pero ésto no se cumple todos los años: hacen falta más datos de otros años para poder corroborar ésto.

% We include the graphics. Year 2007
\begin{figure}[H]
\centering
\begin{subfigure}[H]{0.15\textwidth}
\includegraphics[width=\textwidth]{C:/Users/Faraday/Desktop/practicas_alvaro/maps/media/pm2p5/2007/pm2p5_2007_1_mapa.png}
\captionsetup{labelformat=empty}
\caption{Enero}
\label{fig:map-pm2p5-2007-1}
\end{subfigure}
%
\begin{subfigure}[H]{0.15\textwidth}
\includegraphics[width=\textwidth]{C:/Users/Faraday/Desktop/practicas_alvaro/maps/media/pm2p5/2007/pm2p5_2007_2_mapa.png}
\captionsetup{labelformat=empty}
\caption{Febrero}
\label{fig:map-pm2p5-2007-2}
\end{subfigure}
%
\begin{subfigure}[H]{0.15\textwidth}
\includegraphics[width=\textwidth]{C:/Users/Faraday/Desktop/practicas_alvaro/maps/media/pm2p5/2007/pm2p5_2007_3_mapa.png}
\captionsetup{labelformat=empty}
\caption{Marzo}
\label{fig:map-pm2p5-2007-3}
\end{subfigure}
%
\begin{subfigure}[H]{0.15\textwidth}
\includegraphics[width=\textwidth]{C:/Users/Faraday/Desktop/practicas_alvaro/maps/media/pm2p5/2007/pm2p5_2007_4_mapa.png}
\captionsetup{labelformat=empty}
\caption{Abril}
\label{fig:map-pm2p5-2007-4}
\end{subfigure}
%
\begin{subfigure}[H]{0.15\textwidth}
\includegraphics[width=\textwidth]{C:/Users/Faraday/Desktop/practicas_alvaro/maps/media/pm2p5/2007/pm2p5_2007_5_mapa.png}
\captionsetup{labelformat=empty}
\caption{Mayo}
\label{fig:map-pm2p5-2007-5}
\end{subfigure}
%
\begin{subfigure}[H]{0.15\textwidth}
\includegraphics[width=\textwidth]{C:/Users/Faraday/Desktop/practicas_alvaro/maps/media/pm2p5/2007/pm2p5_2007_6_mapa.png}
\captionsetup{labelformat=empty}
\caption{Junio}
\label{fig:map-pm2p5-2007-6}
\end{subfigure}

\begin{subfigure}[H]{0.15\textwidth}
\includegraphics[width=\textwidth]{C:/Users/Faraday/Desktop/practicas_alvaro/maps/media/pm2p5/2007/pm2p5_2007_7_mapa.png}
\captionsetup{labelformat=empty}
\caption{Julio}
\label{fig:map-pm2p5-2007-7}
\end{subfigure}
%
\begin{subfigure}[H]{0.15\textwidth}
\includegraphics[width=\textwidth]{C:/Users/Faraday/Desktop/practicas_alvaro/maps/media/pm2p5/2007/pm2p5_2007_8_mapa.png}
\captionsetup{labelformat=empty}
\caption{Agosto}
\label{fig:map-pm2p5-2007-8}
\end{subfigure}
%
\begin{subfigure}[H]{0.15\textwidth}
\includegraphics[width=\textwidth]{C:/Users/Faraday/Desktop/practicas_alvaro/maps/media/pm2p5/2007/pm2p5_2007_9_mapa.png}
\captionsetup{labelformat=empty}
\caption{Septiembre}
\label{fig:map-pm2p5-2007-9}
\end{subfigure}
%
\begin{subfigure}[H]{0.15\textwidth}
\includegraphics[width=\textwidth]{C:/Users/Faraday/Desktop/practicas_alvaro/maps/media/pm2p5/2007/pm2p5_2007_10_mapa.png}
\captionsetup{labelformat=empty}
\caption{Octubre}
\label{fig:map-pm2p5-2007-10}
\end{subfigure}
%
\begin{subfigure}[H]{0.15\textwidth}
\includegraphics[width=\textwidth]{C:/Users/Faraday/Desktop/practicas_alvaro/maps/media/pm2p5/2007/pm2p5_2007_11_mapa.png}
\captionsetup{labelformat=empty}
\caption{Noviembre}
\label{fig:map-pm2p5-2007-11}
\end{subfigure}
%
\begin{subfigure}[H]{0.15\textwidth}
\includegraphics[width=\textwidth]{C:/Users/Faraday/Desktop/practicas_alvaro/maps/media/pm2p5/2007/pm2p5_2007_12_mapa.png}
\captionsetup{labelformat=empty}
\caption{Diciembre}
\label{fig:map-pm2p5-2007-12}
\end{subfigure}

\begin{subfigure}[H]{0.45\textwidth}
\includegraphics[width=\textwidth]{C:/Users/Faraday/Desktop/practicas_alvaro/maps/media/pm2p5/color_scale.png}
\captionsetup{labelformat=empty}
\caption{}
\end{subfigure}

\vspace*{-7mm}
\caption{Media mensual de concentración del $PM_{2,5}$ en el año 2007}
\label{fig:map-pm2p5-2007}
\end{figure}

% We include the graphics. Year 2008
\begin{figure}[H]
\centering
\begin{subfigure}[H]{0.15\textwidth}
\includegraphics[width=\textwidth]{C:/Users/Faraday/Desktop/practicas_alvaro/maps/media/pm2p5/2008/pm2p5_2008_1_mapa.png}
\captionsetup{labelformat=empty}
\caption{Enero}
\label{fig:map-pm2p5-2008-1}
\end{subfigure}
%
\begin{subfigure}[H]{0.15\textwidth}
\includegraphics[width=\textwidth]{C:/Users/Faraday/Desktop/practicas_alvaro/maps/media/pm2p5/2008/pm2p5_2008_2_mapa.png}
\captionsetup{labelformat=empty}
\caption{Febrero}
\label{fig:map-pm2p5-2008-2}
\end{subfigure}
%
\begin{subfigure}[H]{0.15\textwidth}
\includegraphics[width=\textwidth]{C:/Users/Faraday/Desktop/practicas_alvaro/maps/media/pm2p5/2008/pm2p5_2008_3_mapa.png}
\captionsetup{labelformat=empty}
\caption{Marzo}
\label{fig:map-pm2p5-2008-3}
\end{subfigure}
%
\begin{subfigure}[H]{0.15\textwidth}
\includegraphics[width=\textwidth]{C:/Users/Faraday/Desktop/practicas_alvaro/maps/media/pm2p5/2008/pm2p5_2008_4_mapa.png}
\captionsetup{labelformat=empty}
\caption{Abril}
\label{fig:map-pm2p5-2008-4}
\end{subfigure}
%
\begin{subfigure}[H]{0.15\textwidth}
\includegraphics[width=\textwidth]{C:/Users/Faraday/Desktop/practicas_alvaro/maps/media/pm2p5/2008/pm2p5_2008_5_mapa.png}
\captionsetup{labelformat=empty}
\caption{Mayo}
\label{fig:map-pm2p5-2008-5}
\end{subfigure}
%
\begin{subfigure}[H]{0.15\textwidth}
\includegraphics[width=\textwidth]{C:/Users/Faraday/Desktop/practicas_alvaro/maps/media/pm2p5/2008/pm2p5_2008_6_mapa.png}
\captionsetup{labelformat=empty}
\caption{Junio}
\label{fig:map-pm2p5-2008-6}
\end{subfigure}

\begin{subfigure}[H]{0.15\textwidth}
\includegraphics[width=\textwidth]{C:/Users/Faraday/Desktop/practicas_alvaro/maps/media/pm2p5/2008/pm2p5_2008_7_mapa.png}
\captionsetup{labelformat=empty}
\caption{Julio}
\label{fig:map-pm2p5-2008-7}
\end{subfigure}
%
\begin{subfigure}[H]{0.15\textwidth}
\includegraphics[width=\textwidth]{C:/Users/Faraday/Desktop/practicas_alvaro/maps/media/pm2p5/2008/pm2p5_2008_8_mapa.png}
\captionsetup{labelformat=empty}
\caption{Agosto}
\label{fig:map-pm2p5-2008-8}
\end{subfigure}
%
\begin{subfigure}[H]{0.15\textwidth}
\includegraphics[width=\textwidth]{C:/Users/Faraday/Desktop/practicas_alvaro/maps/media/pm2p5/2008/pm2p5_2008_9_mapa.png}
\captionsetup{labelformat=empty}
\caption{Septiembre}
\label{fig:map-pm2p5-2008-9}
\end{subfigure}
%
\begin{subfigure}[H]{0.15\textwidth}
\includegraphics[width=\textwidth]{C:/Users/Faraday/Desktop/practicas_alvaro/maps/media/pm2p5/2008/pm2p5_2008_10_mapa.png}
\captionsetup{labelformat=empty}
\caption{Octubre}
\label{fig:map-pm2p5-2008-10}
\end{subfigure}
%
\begin{subfigure}[H]{0.15\textwidth}
\includegraphics[width=\textwidth]{C:/Users/Faraday/Desktop/practicas_alvaro/maps/media/pm2p5/2008/pm2p5_2008_11_mapa.png}
\captionsetup{labelformat=empty}
\caption{Noviembre}
\label{fig:map-pm2p5-2008-11}
\end{subfigure}
%
\begin{subfigure}[H]{0.15\textwidth}
\includegraphics[width=\textwidth]{C:/Users/Faraday/Desktop/practicas_alvaro/maps/media/pm2p5/2008/pm2p5_2008_12_mapa.png}
\captionsetup{labelformat=empty}
\caption{Diciembre}
\label{fig:map-pm2p5-2008-12}
\end{subfigure}

\begin{subfigure}[H]{0.45\textwidth}
\includegraphics[width=\textwidth]{C:/Users/Faraday/Desktop/practicas_alvaro/maps/media/pm2p5/color_scale.png}
\captionsetup{labelformat=empty}
\caption{}
\end{subfigure}

\vspace*{-7mm}
\caption{Media mensual de concentración del $PM_{2,5}$ en el año 2008}
\label{fig:map-pm2p5-2008}
\end{figure}

% We include the graphics. Year 2009
\begin{figure}[H]
\centering
\begin{subfigure}[H]{0.15\textwidth}
\includegraphics[width=\textwidth]{C:/Users/Faraday/Desktop/practicas_alvaro/maps/media/pm2p5/2009/pm2p5_2009_1_mapa.png}
\captionsetup{labelformat=empty}
\caption{Enero}
\label{fig:map-pm2p5-2009-1}
\end{subfigure}
%
\begin{subfigure}[H]{0.15\textwidth}
\includegraphics[width=\textwidth]{C:/Users/Faraday/Desktop/practicas_alvaro/maps/media/pm2p5/2009/pm2p5_2009_2_mapa.png}
\captionsetup{labelformat=empty}
\caption{Febrero}
\label{fig:map-pm2p5-2009-2}
\end{subfigure}
%
\begin{subfigure}[H]{0.15\textwidth}
\includegraphics[width=\textwidth]{C:/Users/Faraday/Desktop/practicas_alvaro/maps/media/pm2p5/2009/pm2p5_2009_3_mapa.png}
\captionsetup{labelformat=empty}
\caption{Marzo}
\label{fig:map-pm2p5-2009-3}
\end{subfigure}
%
\begin{subfigure}[H]{0.15\textwidth}
\includegraphics[width=\textwidth]{C:/Users/Faraday/Desktop/practicas_alvaro/maps/media/pm2p5/2009/pm2p5_2009_4_mapa.png}
\captionsetup{labelformat=empty}
\caption{Abril}
\label{fig:map-pm2p5-2009-4}
\end{subfigure}
%
\begin{subfigure}[H]{0.15\textwidth}
\includegraphics[width=\textwidth]{C:/Users/Faraday/Desktop/practicas_alvaro/maps/media/pm2p5/2009/pm2p5_2009_5_mapa.png}
\captionsetup{labelformat=empty}
\caption{Mayo}
\label{fig:map-pm2p5-2009-5}
\end{subfigure}
%
\begin{subfigure}[H]{0.15\textwidth}
\includegraphics[width=\textwidth]{C:/Users/Faraday/Desktop/practicas_alvaro/maps/media/pm2p5/2009/pm2p5_2009_6_mapa.png}
\captionsetup{labelformat=empty}
\caption{Junio}
\label{fig:map-pm2p5-2009-6}
\end{subfigure}

\begin{subfigure}[H]{0.15\textwidth}
\includegraphics[width=\textwidth]{C:/Users/Faraday/Desktop/practicas_alvaro/maps/media/pm2p5/2009/pm2p5_2009_7_mapa.png}
\captionsetup{labelformat=empty}
\caption{Julio}
\label{fig:map-pm2p5-2009-7}
\end{subfigure}
%
\begin{subfigure}[H]{0.15\textwidth}
\includegraphics[width=\textwidth]{C:/Users/Faraday/Desktop/practicas_alvaro/maps/media/pm2p5/2009/pm2p5_2009_8_mapa.png}
\captionsetup{labelformat=empty}
\caption{Agosto}
\label{fig:map-pm2p5-2009-8}
\end{subfigure}
%
\begin{subfigure}[H]{0.15\textwidth}
\includegraphics[width=\textwidth]{C:/Users/Faraday/Desktop/practicas_alvaro/maps/media/pm2p5/2009/pm2p5_2009_9_mapa.png}
\captionsetup{labelformat=empty}
\caption{Septiembre}
\label{fig:map-pm2p5-2009-9}
\end{subfigure}
%
\begin{subfigure}[H]{0.15\textwidth}
\includegraphics[width=\textwidth]{C:/Users/Faraday/Desktop/practicas_alvaro/maps/media/pm2p5/2009/pm2p5_2009_10_mapa.png}
\captionsetup{labelformat=empty}
\caption{Octubre}
\label{fig:map-pm2p5-2009-10}
\end{subfigure}
%
\begin{subfigure}[H]{0.15\textwidth}
\includegraphics[width=\textwidth]{C:/Users/Faraday/Desktop/practicas_alvaro/maps/media/pm2p5/2009/pm2p5_2009_11_mapa.png}
\captionsetup{labelformat=empty}
\caption{Noviembre}
\label{fig:map-pm2p5-2009-11}
\end{subfigure}
%
\begin{subfigure}[H]{0.15\textwidth}
\includegraphics[width=\textwidth]{C:/Users/Faraday/Desktop/practicas_alvaro/maps/media/pm2p5/2009/pm2p5_2009_12_mapa.png}
\captionsetup{labelformat=empty}
\caption{Diciembre}
\label{fig:map-pm2p5-2009-12}
\end{subfigure}

\begin{subfigure}[H]{0.45\textwidth}
\includegraphics[width=\textwidth]{C:/Users/Faraday/Desktop/practicas_alvaro/maps/media/pm2p5/color_scale.png}
\captionsetup{labelformat=empty}
\caption{}
\end{subfigure}

\vspace*{-7mm}
\caption{Media mensual de concentración del $PM_{2,5}$ en el año 2009}
\label{fig:map-pm2p5-2009}
\end{figure}

% We include the graphics. Year 2010
\begin{figure}[H]
\centering
\begin{subfigure}[H]{0.15\textwidth}
\includegraphics[width=\textwidth]{C:/Users/Faraday/Desktop/practicas_alvaro/maps/media/pm2p5/2010/pm2p5_2010_1_mapa.png}
\captionsetup{labelformat=empty}
\caption{Enero}
\label{fig:map-pm2p5-2010-1}
\end{subfigure}
%
\begin{subfigure}[H]{0.15\textwidth}
\includegraphics[width=\textwidth]{C:/Users/Faraday/Desktop/practicas_alvaro/maps/media/pm2p5/2010/pm2p5_2010_2_mapa.png}
\captionsetup{labelformat=empty}
\caption{Febrero}
\label{fig:map-pm2p5-2010-2}
\end{subfigure}
%
\begin{subfigure}[H]{0.15\textwidth}
\includegraphics[width=\textwidth]{C:/Users/Faraday/Desktop/practicas_alvaro/maps/media/pm2p5/2010/pm2p5_2010_3_mapa.png}
\captionsetup{labelformat=empty}
\caption{Marzo}
\label{fig:map-pm2p5-2010-3}
\end{subfigure}
%
\begin{subfigure}[H]{0.15\textwidth}
\includegraphics[width=\textwidth]{C:/Users/Faraday/Desktop/practicas_alvaro/maps/media/pm2p5/2010/pm2p5_2010_4_mapa.png}
\captionsetup{labelformat=empty}
\caption{Abril}
\label{fig:map-pm2p5-2010-4}
\end{subfigure}
%
\begin{subfigure}[H]{0.15\textwidth}
\includegraphics[width=\textwidth]{C:/Users/Faraday/Desktop/practicas_alvaro/maps/media/pm2p5/2010/pm2p5_2010_5_mapa.png}
\captionsetup{labelformat=empty}
\caption{Mayo}
\label{fig:map-pm2p5-2010-5}
\end{subfigure}
%
\begin{subfigure}[H]{0.15\textwidth}
\includegraphics[width=\textwidth]{C:/Users/Faraday/Desktop/practicas_alvaro/maps/media/pm2p5/2010/pm2p5_2010_6_mapa.png}
\captionsetup{labelformat=empty}
\caption{Junio}
\label{fig:map-pm2p5-2010-6}
\end{subfigure}

\begin{subfigure}[H]{0.15\textwidth}
\includegraphics[width=\textwidth]{C:/Users/Faraday/Desktop/practicas_alvaro/maps/media/pm2p5/2010/pm2p5_2010_7_mapa.png}
\captionsetup{labelformat=empty}
\caption{Julio}
\label{fig:map-pm2p5-2010-7}
\end{subfigure}
%
\begin{subfigure}[H]{0.15\textwidth}
\includegraphics[width=\textwidth]{C:/Users/Faraday/Desktop/practicas_alvaro/maps/media/pm2p5/2010/pm2p5_2010_8_mapa.png}
\captionsetup{labelformat=empty}
\caption{Agosto}
\label{fig:map-pm2p5-2010-8}
\end{subfigure}
%
\begin{subfigure}[H]{0.15\textwidth}
\includegraphics[width=\textwidth]{C:/Users/Faraday/Desktop/practicas_alvaro/maps/media/pm2p5/2010/pm2p5_2010_9_mapa.png}
\captionsetup{labelformat=empty}
\caption{Septiembre}
\label{fig:map-pm2p5-2010-9}
\end{subfigure}
%
\begin{subfigure}[H]{0.15\textwidth}
\includegraphics[width=\textwidth]{C:/Users/Faraday/Desktop/practicas_alvaro/maps/media/pm2p5/2010/pm2p5_2010_10_mapa.png}
\captionsetup{labelformat=empty}
\caption{Octubre}
\label{fig:map-pm2p5-2010-10}
\end{subfigure}
%
\begin{subfigure}[H]{0.15\textwidth}
\includegraphics[width=\textwidth]{C:/Users/Faraday/Desktop/practicas_alvaro/maps/media/pm2p5/2010/pm2p5_2010_11_mapa.png}
\captionsetup{labelformat=empty}
\caption{Noviembre}
\label{fig:map-pm2p5-2010-11}
\end{subfigure}
%
\begin{subfigure}[H]{0.15\textwidth}
\includegraphics[width=\textwidth]{C:/Users/Faraday/Desktop/practicas_alvaro/maps/media/pm2p5/2010/pm2p5_2010_12_mapa.png}
\captionsetup{labelformat=empty}
\caption{Diciembre}
\label{fig:map-pm2p5-2010-12}
\end{subfigure}

\begin{subfigure}[H]{0.45\textwidth}
\includegraphics[width=\textwidth]{C:/Users/Faraday/Desktop/practicas_alvaro/maps/media/pm2p5/color_scale.png}
\captionsetup{labelformat=empty}
\caption{}
\end{subfigure}

\vspace*{-7mm}
\caption{Media mensual de concentración del $PM_{2,5}$ en el año 2010}
\label{fig:map-pm2p5-2010}
\end{figure}

% We include the graphics. Year 2011
\begin{figure}[H]
\centering
\begin{subfigure}[H]{0.15\textwidth}
\includegraphics[width=\textwidth]{C:/Users/Faraday/Desktop/practicas_alvaro/maps/media/pm2p5/2011/pm2p5_2011_1_mapa.png}
\captionsetup{labelformat=empty}
\caption{Enero}
\label{fig:map-pm2p5-2011-1}
\end{subfigure}
%
\begin{subfigure}[H]{0.15\textwidth}
\includegraphics[width=\textwidth]{C:/Users/Faraday/Desktop/practicas_alvaro/maps/media/pm2p5/2011/pm2p5_2011_2_mapa.png}
\captionsetup{labelformat=empty}
\caption{Febrero}
\label{fig:map-pm2p5-2011-2}
\end{subfigure}
%
\begin{subfigure}[H]{0.15\textwidth}
\includegraphics[width=\textwidth]{C:/Users/Faraday/Desktop/practicas_alvaro/maps/media/pm2p5/2011/pm2p5_2011_3_mapa.png}
\captionsetup{labelformat=empty}
\caption{Marzo}
\label{fig:map-pm2p5-2011-3}
\end{subfigure}
%
\begin{subfigure}[H]{0.15\textwidth}
\includegraphics[width=\textwidth]{C:/Users/Faraday/Desktop/practicas_alvaro/maps/media/pm2p5/2011/pm2p5_2011_4_mapa.png}
\captionsetup{labelformat=empty}
\caption{Abril}
\label{fig:map-pm2p5-2011-4}
\end{subfigure}
%
\begin{subfigure}[H]{0.15\textwidth}
\includegraphics[width=\textwidth]{C:/Users/Faraday/Desktop/practicas_alvaro/maps/media/pm2p5/2011/pm2p5_2011_5_mapa.png}
\captionsetup{labelformat=empty}
\caption{Mayo}
\label{fig:map-pm2p5-2011-5}
\end{subfigure}
%
\begin{subfigure}[H]{0.15\textwidth}
\includegraphics[width=\textwidth]{C:/Users/Faraday/Desktop/practicas_alvaro/maps/media/pm2p5/2011/pm2p5_2011_6_mapa.png}
\captionsetup{labelformat=empty}
\caption{Junio}
\label{fig:map-pm2p5-2011-6}
\end{subfigure}

\begin{subfigure}[H]{0.15\textwidth}
\includegraphics[width=\textwidth]{C:/Users/Faraday/Desktop/practicas_alvaro/maps/media/pm2p5/2011/pm2p5_2011_7_mapa.png}
\captionsetup{labelformat=empty}
\caption{Julio}
\label{fig:map-pm2p5-2011-7}
\end{subfigure}
%
\begin{subfigure}[H]{0.15\textwidth}
\includegraphics[width=\textwidth]{C:/Users/Faraday/Desktop/practicas_alvaro/maps/media/pm2p5/2011/pm2p5_2011_8_mapa.png}
\captionsetup{labelformat=empty}
\caption{Agosto}
\label{fig:map-pm2p5-2011-8}
\end{subfigure}
%
\begin{subfigure}[H]{0.15\textwidth}
\includegraphics[width=\textwidth]{C:/Users/Faraday/Desktop/practicas_alvaro/maps/media/pm2p5/2011/pm2p5_2011_9_mapa.png}
\captionsetup{labelformat=empty}
\caption{Septiembre}
\label{fig:map-pm2p5-2011-9}
\end{subfigure}
%
\begin{subfigure}[H]{0.15\textwidth}
\includegraphics[width=\textwidth]{C:/Users/Faraday/Desktop/practicas_alvaro/maps/media/pm2p5/2011/pm2p5_2011_10_mapa.png}
\captionsetup{labelformat=empty}
\caption{Octubre}
\label{fig:map-pm2p5-2011-10}
\end{subfigure}
%
\begin{subfigure}[H]{0.15\textwidth}
\includegraphics[width=\textwidth]{C:/Users/Faraday/Desktop/practicas_alvaro/maps/media/pm2p5/2011/pm2p5_2011_11_mapa.png}
\captionsetup{labelformat=empty}
\caption{Noviembre}
\label{fig:map-pm2p5-2011-11}
\end{subfigure}
%
\begin{subfigure}[H]{0.15\textwidth}
\includegraphics[width=\textwidth]{C:/Users/Faraday/Desktop/practicas_alvaro/maps/media/pm2p5/2011/pm2p5_2011_12_mapa.png}
\captionsetup{labelformat=empty}
\caption{Diciembre}
\label{fig:map-pm2p5-2011-12}
\end{subfigure}

\begin{subfigure}[H]{0.45\textwidth}
\includegraphics[width=\textwidth]{C:/Users/Faraday/Desktop/practicas_alvaro/maps/media/pm2p5/color_scale.png}
\captionsetup{labelformat=empty}
\caption{}
\end{subfigure}

\vspace*{-7mm}
\caption{Media mensual de concentración del $PM_{2,5}$ en el año 2011}
\label{fig:map-pm2p5-2011}
\end{figure}

Finalmente, respecto a la distribución anual de $PM_{2,5}$ en Inglaterra y Gales, parece que hay tendencia en haber menos concentración en Gales (sobre todo en el este) y el norte de Inglaterra, aunque no se puede afirmar con total rotundidad ésto. Ahora bien, durante los cinco años que se han analizado, casi todo el territorio de Inglaterra y Gales ha cumplido la normativa de la UE respecto a la concentración de $PM_{2,5}$ (máximo de 25 $\nicefrac{\mu g}{m^3}$ de media anual). Sin embargo, casi todo el territorio no ha cumplido el margen establecido por la OMS (con un máximo de 10 $\nicefrac{\mu g}{m^3}$ de media anual). Todo ésto puede comprobarse en la figura \ref{fig:map-pm2p5-anual} y en el cuadro \ref{tab:annual_pm2p5}.

% We include the annual concentration maps
\begin{figure}[H]
\centering
\begin{subfigure}[H]{0.18\textwidth}
\includegraphics[width=\textwidth]{C:/Users/Faraday/Desktop/practicas_alvaro/maps/media/pm2p5/anual/pm2p5_2007_anual_mapa.png}
\captionsetup{labelformat=empty}
\caption{2007}
\end{subfigure}
%
\begin{subfigure}[H]{0.18\textwidth}
\includegraphics[width=\textwidth]{C:/Users/Faraday/Desktop/practicas_alvaro/maps/media/pm2p5/anual/pm2p5_2008_anual_mapa.png}
\captionsetup{labelformat=empty}
\caption{2008}
\end{subfigure}
%
\begin{subfigure}[H]{0.18\textwidth}
\includegraphics[width=\textwidth]{C:/Users/Faraday/Desktop/practicas_alvaro/maps/media/pm2p5/anual/pm2p5_2009_anual_mapa.png}
\captionsetup{labelformat=empty}
\caption{2009}
\end{subfigure}
%
\begin{subfigure}[H]{0.18\textwidth}
\includegraphics[width=\textwidth]{C:/Users/Faraday/Desktop/practicas_alvaro/maps/media/pm2p5/anual/pm2p5_2010_anual_mapa.png}
\captionsetup{labelformat=empty}
\caption{2010}
\end{subfigure}
%
\begin{subfigure}[H]{0.18\textwidth}
\includegraphics[width=\textwidth]{C:/Users/Faraday/Desktop/practicas_alvaro/maps/media/pm2p5/anual/pm2p5_2011_anual_mapa.png}
\captionsetup{labelformat=empty}
\caption{2011}
\end{subfigure}

\begin{subfigure}[H]{0.45\textwidth}
\includegraphics[width=\textwidth]{C:/Users/Faraday/Desktop/practicas_alvaro/maps/media/pm2p5/color_scale.png}
\captionsetup{labelformat=empty}
\caption{}
\end{subfigure}

\vspace*{-7mm}
\caption{Media anual de concentración del $PM_{2,5}$ en el período 2007-2011}
\label{fig:map-pm2p5-anual}
\end{figure}

% Table
\begin{table}[H]
\centering
\begin{tabularx}{\textwidth}{|c| *{3}{>{\centering\arraybackslash}X|}}
\hline
 & \multicolumn{3}{c|}{Porcentaje de la superficie de Inglaterra y Gales contaminada de acuerdo} \\
 Año & \multicolumn{3}{c|}{al rango de concentración de $PM_{2,5}$ ($\frac{\mu g}{m^{3}}$)} \\ \cline{2-4}
  & (0, 10) & (10, 20) & (20, $\infty$) \\
 \hline
 2007 & 0,00 & 99,90 & 0,10 \\
 \hline
 2008 & 0,00 & 99,57 & 0,43 \\
 \hline
 2009 & 0,0 & 100,00 & 0,00 \\
 \hline
 2010 & 0,0 & 100,00 & 0,00 \\
 \hline
 2011 & 0,0 & 100,00 & 0,00 \\
 \hline
\end{tabularx}
\caption{}
\label{tab:annual_pm2p5}
\end{table}

\newpage

\section{Conclusiones}

Tras el análisis de los resultados obtenidos, hemos podido analizar y explicar las concentraciones de $NO_{2}$ y de $O_{3}$ 

\newpage

\section{Futuras líneas de estudio}

Introducir aquí las futuras líneas de estudio

\newpage

\section{Bibliografía}

\begin{thebibliography}{99}

\bibitem{ozone_spain} Díaz, J., Ortiz, C., Falcón, I., Salvador, C. y Linares, C. (2018). Short-term effect of tropospheric ozone on daily mortality in Spain. \textit{Atmospheric environment}, 187: 107-116.

\bibitem{co2_cycles} Fernández-Duque, B., Pérez, I. A., García, M. A., Pardo, N. y Sánchez, M. L. (2019). Annual and seasonal cycles of $CO_{2}$ and $CH_{4}$ in a Mediterranean Spanish environment using different kernel functions. \textit{Stochastic Environmental Research and Risk Assessment}, 33: 915-930.

\bibitem{directiva_50_ce} Parlamento Europeo y Consejo de la Unión Europea (2008). Directiva 2008/50/CE del Parlamento Europeo y del Consejo, de 21 de mayo de 2008, relativa a la calidad del aire ambiente y a una atmósfera más limpia en Europa. \textit{Diario Oficial de la Unión Europea}, 152: 1-44.

\bibitem{pollution_tarragona} Rovira, J., Domingo, J. L. y Schuhmacher, M. (2020). Air quality, health impacts and burden of disease due to air pollution (PM10, PM2.5, NO2 and O3): Application of AirQ+ model to the Camp de Tarragona County (Catalonia, Spain). \textit{Science of the total environment}, 703.

\bibitem{pm_effects_healthy} Shaughnessy, W. J., Venigalla, M. M. y Trump, D. (2015). Health effects of ambient levels of respirable particulate matter (PM) on healthy, young-adult population. \textit{Atmospheric environment}, 123: 102-111.

\bibitem{statistical_book} Wallace, J.M y Hobbs, P.V (2006). Atmospheric Chemistry. En: \textit{Atmospheric Science. An Introductory Survey, Second Edition.}. Academic Press. 153-207.

\bibitem{who_guidelines} WHO (2006). Air quality guidelines. Global update 2005. Particulate matter, ozone, nitrogen dioxide and sulfur dioxide. \textit{World Health Organization}.

\bibitem{who_guidelines_summary} WHO (2006). Guías de calidad del aire de la OMS relativas al material particulado, el ozono, el dióxido de nitrógeno y el dióxido de azufre. Actualización mundial 2005. Resumen de evaluación de los riesgos. \textit{World Health Organization}.

\bibitem{statistical_book} Wilks, D.S (2020). Empirical Distributions and Exploratory Data Analysis. En: \textit{Statistical Methods in the Atmospheric Sciences, Fourth edition.}. Elsevier. 23-75.

\end{thebibliography}

\end{document}
