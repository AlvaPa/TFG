%Template created in order to write a scientific document from an university or enterprise in spanish.

%We initiate the packages and functions
\documentclass[12pt]{article}
\usepackage{geometry}
\geometry{a4paper,
total={170mm, 257mm},
left=20mm,
top=20mm
}
\usepackage[utf8]{inputenc}
\usepackage[T1]{fontenc}
\usepackage[spanish]{babel}
\usepackage{mathptmx}
\usepackage{amsmath}
\usepackage{amssymb}
\usepackage{graphicx}

\begin{document}

%Title page
\begin{titlepage}

%Corporation's logo
\begin{center}

\begin{minipage}[c]{170mm}
\includegraphics[scale=0.7]{logo_uva}
\hspace{3cm}
\includegraphics[scale=0.4]{logotipo-3-3}

\hspace{1.2cm}Facultad de ciencias
\end{minipage}

\end{center}
%Title, author and date
\begin{center}

\vspace{4cm}
\Huge \textbf{TRABAJO FIN DE GRADO} \\
\vspace{2cm}
\Large Grado en Física \\
\vspace{2cm}
{\Large \textbf{Análisis espacio-temporal de las concentraciones de distintos contaminantes}} \\
\vspace{4cm}
\normalsize \textbf{Autor: Álvaro Patricio Prieto Pérez} \\
\vspace{1cm}
\textbf{Tutor/es: Isidro Pérez Bartolomé \\ María Ángeles García}

\end{center}

\end{titlepage}

%Giving thanks
\newpage
\vspace{10cm}
\newpage
\Huge Agradecimientos

\vspace{0.5cm}
\normalsize En primer lugar, quiero agradecer a mis tutores por haberme ayudado y ofrecido sus puntos de vista para la realización de este trabajo, así como a la European Centre for Environment and Human Health por los datos disponibles. Del mismo modo, quiero agradecer a todos mis familiares y amigos que han estado allí tanto en los momentos buenos como en los malos, pues el apoyo que me han dado a lo largo de mi vida ha supuesto que este trabajo sea posible.

%Table of contents
\newpage
\tableofcontents
\newpage

%We start the document from here
\section{Una breve introducción} 

Jejejeje

\end{document}